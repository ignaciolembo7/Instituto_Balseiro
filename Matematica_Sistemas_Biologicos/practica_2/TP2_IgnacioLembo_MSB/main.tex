%%%%%%%%%%%%%%%%%%%%%%%%%%%%%%%%%%%%%%%%%
% 
% LaTeX tufte book
% Version 2.0 (12/02/21)
%
% Original authors:
% Ignacio Lembo Ferrari(lemboignacio98@gmail.com)
%%%%%%%%%%%%%%%%%%%%%%%%%%%%%%%%%%%%%%%%

%% Unfortunately for the contents to contain
%% the "Parts" lines successfully, hyperref
%% needs to be disabled.
\documentclass[a4paper,nohyper,nobib]{tufte-book}
%%%%%%%%%%%%%%%%%%%%%%%%%%%%%%%%%%%%%%%%%
\usepackage{booktabs}
\usepackage[utf8]{inputenc}
\usepackage[spanish]{babel}
\usepackage{csquotes}
\usepackage{tikz}
\usepackage{fancyhdr}
\usepackage{blindtext}
\usepackage{subcaption}
\usepackage{hyperref}
\usepackage[shortlabels]{enumitem}
\usepackage{caption,amsfonts,epsfig,amsmath,amssymb,amsbsy,bm,multicol}
\usepackage{multirow}
\usepackage{media9}
%\usepackage[siunitx]{circuitikz}
%\usepackage[amsmath]{empheq}
\usepackage{epigraph}
\usepackage[T1]{fontenc}%%
\usepackage[shortlabels]{enumitem}
%\usepackage{verbatim}

\usepackage{nameref}
% \hypersetup{colorlinks}% uncomment this line if you prefer colored hyperlinks (e.g., for onscreen viewing)

% \usepackage{hyphenat}
\usepackage{url}
\usepackage[backend=biber, natbib=true, style=numeric]{biblatex}
\addbibresource{sample-handout.bib}
\usepackage{xargs}
\renewcommandx{\cite}[3][1={0pt},2={}]{\sidenote[][#1]{\fullcite[#2]{#3}}}

%%%%%%%%%%%%%%%%%%%%%%%%%%%%%%%%%%%%%%%%%%%
% Book metadata
\title{Modelos de una sola población}
\date{Relatividad General}
\author[Ignacio Lembo Ferrari]{Ignacio Lembo Ferrari}
\publisher[]{UNIVERSIDAD NACIONAL DE ROSARIO\\
\bigskip
FACULTAD DE CIENCIAS EXACTAS, INGENIERÍA Y AGRIMENSURA\\
\bigskip
DEPARTAMENTO DE FÍSICA-ESCUELA DE CIENCIAS EXACTAS Y NATURALES
}

%%%%%%%%%%%%%%%%%%%%%%%%%%%%%%%%%%%%%%%%%%%%%%
%\usepackage{microtype}

%%
% For nicely typeset tabular material
%\usepackage{booktabs}

%%
% For graphics / images
\usepackage{graphicx}
\setkeys{Gin}{width=\linewidth,totalheight=\textheight,keepaspectratio}
\graphicspath{{graphics/}}



% Prints argument within hanging parentheses (i.e., parentheses that take
% up no horizontal space).  Useful in tabular environments.
\newcommand{\hangp}[1]{\makebox[0pt][r]{(}#1\makebox[0pt][l]{)}}

%%
% Prints an asterisk that takes up no horizontal space.
% Useful in tabular environments.
\newcommand{\hangstar}{\makebox[0pt][l]{*}}

%%
% Prints a trailing space in a smart way.
\usepackage{xspace}

%%
% Some shortcuts for Tufte's book titles.  The lowercase commands will
% produce the initials of the book title in italics.  The all-caps commands
% will print out the full title of the book in italics.
%\newcommand{\TL}{Tufte-\LaTeX\xspace}

% Prints an epigraph and speaker in sans serif, all-caps type.
\newcommand{\openepigraph}[2]{%
  %\sffamily\fontsize{14}{16}\selectfont
  \begin{fullwidth}
  \sffamily\large
  \begin{doublespace}
  \noindent\allcaps{#1}\\% epigraph
  \noindent\allcaps{#2}% author
  \end{doublespace}
  \end{fullwidth}
}

% Inserts a blank page
\newcommand{\blankpage}{\newpage\hbox{}\thispagestyle{empty}\newpage}

\usepackage{units}

% Typesets the font size, leading, and measure in the form of 10/12x26 pc.
\newcommand{\measure}[3]{#1/#2$\times$\unit[#3]{pc}}

% Macros for typesetting the documentation
\newcommand{\hlred}[1]{\textcolor{Maroon}{#1}}% prints in red
\newcommand{\hangleft}[1]{\makebox[0pt][r]{#1}}
\newcommand{\hairsp}{\hspace{1pt}}% hair space
\newcommand{\hquad}{\hskip0.5em\relax}% half quad space
\newcommand{\TODO}{\textcolor{red}{\bf !TODO}\xspace}
\newcommand{\ie}{\textit{i.\hairsp{}e.}\xspace}
\newcommand{\eg}{\textit{e.\hairsp{}g.}\xspace}
\newcommand{\na}{\quad--}% used in tables for N/A cells
\providecommand{\XeLaTeX}{X\lower.5ex\hbox{\kern-0.15em\reflectbox{E}}\kern-0.1em\LaTeX}
\newcommand{\tXeLaTeX}{\XeLaTeX\index{XeLaTeX@\protect\XeLaTeX}}
% \index{\texttt{\textbackslash xyz}@\hangleft{\texttt{\textbackslash}}\texttt{xyz}}
\newcommand{\tuftebs}{\symbol{'134}}% a backslash in tt type in OT1/T1
\newcommand{\doccmdnoindex}[2][]{\texttt{\tuftebs#2}}% command name -- adds backslash automatically (and doesn't add cmd to the index)
\newcommand{\doccmddef}[2][]{%
  \hlred{\texttt{\tuftebs#2}}\label{cmd:#2}%
  \ifthenelse{\isempty{#1}}%
    {% add the command to the index
      \index{#2 command@\protect\hangleft{\texttt{\tuftebs}}\texttt{#2}}% command name
    }%
    {% add the command and package to the index
      \index{#2 command@\protect\hangleft{\texttt{\tuftebs}}\texttt{#2} (\texttt{#1} package)}% command name
      \index{#1 package@\texttt{#1} package}\index{packages!#1@\texttt{#1}}% package name
    }%
}% command name -- adds backslash automatically
\newcommand{\doccmd}[2][]{%
  \texttt{\tuftebs#2}%
  \ifthenelse{\isempty{#1}}%
    {% add the command to the index
      \index{#2 command@\protect\hangleft{\texttt{\tuftebs}}\texttt{#2}}% command name
    }%
    {% add the command and package to the index
      \index{#2 command@\protect\hangleft{\texttt{\tuftebs}}\texttt{#2} (\texttt{#1} package)}% command name
      \index{#1 package@\texttt{#1} package}\index{packages!#1@\texttt{#1}}% package name
    }%
}% command name -- adds backslash automatically
\newcommand{\docopt}[1]{\ensuremath{\langle}\textrm{\textit{#1}}\ensuremath{\rangle}}% optional command argument
\newcommand{\docarg}[1]{\textrm{\textit{#1}}}% (required) command argument
\newenvironment{docspec}{\begin{quotation}\ttfamily\parskip0pt\parindent0pt\ignorespaces}{\end{quotation}}% command specification environment
\newcommand{\docenv}[1]{\texttt{#1}\index{#1 environment@\texttt{#1} environment}\index{environments!#1@\texttt{#1}}}% environment name
\newcommand{\docenvdef}[1]{\hlred{\texttt{#1}}\label{env:#1}\index{#1 environment@\texttt{#1} environment}\index{environments!#1@\texttt{#1}}}% environment name
\newcommand{\docpkg}[1]{\texttt{#1}\index{#1 package@\texttt{#1} package}\index{packages!#1@\texttt{#1}}}% package name
\newcommand{\doccls}[1]{\texttt{#1}}% document class name
\newcommand{\docclsopt}[1]{\texttt{#1}\index{#1 class option@\texttt{#1} class option}\index{class options!#1@\texttt{#1}}}% document class option name
\newcommand{\docclsoptdef}[1]{\hlred{\texttt{#1}}\label{clsopt:#1}\index{#1 class option@\texttt{#1} class option}\index{class options!#1@\texttt{#1}}}% document class option name defined
\newcommand{\docmsg}[2]{\bigskip\begin{fullwidth}\noindent\ttfamily#1\end{fullwidth}\medskip\par\noindent#2}
\newcommand{\docfilehook}[2]{\texttt{#1}\index{file hooks!#2}\index{#1@\texttt{#1}}}
\newcommand{\doccounter}[1]{\texttt{#1}\index{#1 counter@\texttt{#1} counter}}

% Generates the index
\usepackage{makeidx}
\makeindex

%%%% Kevin Godny's code for title page and contents from https://groups.google.com/forum/#!topic/tufte-latex/ujdzrktC1BQ
%\usepackage{pdfpages}
%\makeatletter
%\renewcommand{\maketitlepage}{%
%\begingroup%
%\includepdf[pages=1,fitpaper]{cover1}
%\thispagestyle{empty}
%\endgroup
%}
\makeatother
%%%%%%%%%%%%%%%%%%%%%%%%%%%%%%%%%%%
%COPYRIGHT - Franco Cassinese, Angel Ramírez & Ignacio Lembo Ferrari UNR% 
%%%%%%%%%%%%%%%%%%%%%%%%%%%%%%%%%%%%
%Cajas de colores
\usepackage[many]{tcolorbox}
\usetikzlibrary{calc}

%Definicion de colores
\definecolor{myred}{RGB}{203,72,41}
\definecolor{mygreen}{rgb}{0.55,0.95,0.70}
\definecolor{darkgreen}{RGB}{1,108,10}

\tcbset{mystyle/.style={
  breakable,
  enhanced,
  outer arc=0pt,
  arc=0pt,
  colframe=darkgreen,
  colback=mygreen,
  coltitle=black,
  attach boxed title to top left,
  boxed title style={
    colback=mygreen,
    outer arc=0pt,
    arc=0pt,
    top=3pt,
    bottom=2pt,
    },
  fonttitle=\sffamily
  }
}
%%Cajas para resultados importantes
\newtcolorbox[auto counter]{resultados}[1][]{
  mystyle,
  title=Resultado~\thetcbcounter,
  overlay unbroken and first={
      \path
        let
        \p1=(title.north east),
        \p2=(frame.north east)
        in
        node[anchor=west,font=\sffamily,color=mygreen,text width=\x2-\x1] 
        at (title.east) {#1};
  }
}
%%Cajas para remarcar cosas que son importantes porque pueden servir despues o son conceptos que queremos que queden
\newtcolorbox{remarkbox}[2][]{
    title=Remark Box~\thetcbcounter,
    lower separated=false,
    colback=myred!20,
    colframe=myred,fonttitle=\bfseries,
    colbacktitle=myred!80,
    coltitle=black,
    enhanced,
    attach boxed title to top left=
    {xshift=0.5cm,yshift=-2mm},
    title=#2,#1
}
%%Cajas para añadir data extra que puede ser opcional de leer
\newtcolorbox{extrabox}[2][]{
    title=Extradata~\thetcbcounter,
    lower separated=false,
    colback=myred!50,
    arc=0pt,
    colframe=myred,fonttitle=\bfseries,
    colbacktitle=myred!50,
    coltitle=white,
    enhanced,
    attach boxed title to top left=
    {xshift=0.5cm,yshift=-2mm},
    title=#2,#1
}
%%%%%%%%%%%%%%%%%%%%%%%%%%%%%%%%%%%%%%%
%Chapter style
\usepackage{xcolor} % for colour
\setcounter{secnumdepth}{2}

%Options: Sonny, Lenny, Glenn, Conny, Rejne, Bjarne, Bjornstrup
\usepackage[T1]{fontenc}
\usepackage[Bjornstrup]{fncychap}
\ChNumVar{\fontsize{76}{80}\usefont{OT1}{ptm}{m}{n}\selectfont\textcolor{myred}}
\ChTitleVar{\raggedleft\Huge\sffamily\bfseries}


%%%%%%%%%%%%%%%%%%%%%%%%%%%%%%%%%%%%%%%%%
%Section style
\def\chpcolor{myred!90}
\def\chpcolortxt{myred!90}
\def\sectionfont{\sffamily\LARGE}
\makeatletter

\def\@sectionstrut{\vrule\@width\z@\@height12.5\p@}
\def\@makesectionhead#1{%
  {\par\vspace{20pt}%
   \parindent 0pt\raggedleft\sectionfont
   \colorbox{\chpcolor}{%
     \parbox[t]{90pt}{\color{white}\@sectionstrut\@depth4.5\p@\hfill
       \ifnum\c@secnumdepth>\z@\thesection\fi}%
   }%
   \begin{minipage}[t]{\dimexpr\textwidth-90pt-2\fboxsep\relax}
   \color{\chpcolortxt}\@sectionstrut\hspace{5pt}#1
   \end{minipage}\par
   \vspace{10pt}%
  }
}
\def\section{\@afterindentfalse\secdef\@section\@ssection}
\def\@section[#1]#2{%
  \ifnum\c@secnumdepth>\m@ne
    \refstepcounter{section}%
    \addcontentsline{toc}{section}{\protect\numberline{\thesection}#1}%
  \else
    \phantomsection
    \addcontentsline{toc}{section}{#1}%
  \fi
  \sectionmark{#1}%
  \if@twocolumn
    \@topnewpage[\@makesectionhead{#2}]%
  \else
    \@makesectionhead{#2}\@afterheading
  \fi
}
\def\@ssection#1{%
  \if@twocolumn
    \@topnewpage[\@makesectionhead{#1}]%
  \else
    \@makesectionhead{#1}\@afterheading
  \fi
}
\makeatother
%%%%%%%%%%%%%%%%%%%%%%%%%%%%%%%%%%%%%%%%%%%%%%
\hypersetup{
    colorlinks=true,
    linkcolor=red,
    filecolor=magenta,      
    urlcolor=violet,
    citecolor=magenta,
}

%----------------------------------------------------------------------------------------
%	BEGINNING OF DOCUMENT
%----------------------------------------------------------------------------------------

\begin{document}

\frontmatter

\maketitle

\mainmatter

\input{Chapter1a}%Review of GR 
%\chapter{\textcolor{myred}{Ecuaciones de Campo de Einstein}}

\newthought{La validez} de la teoría gravitacional de Newton es indiscutible. Sobre las bases de esta teoría podemos determinar el movimiento de planetas, satélites, naves espaciales y hasta fuimos capaces de predecir la existencia de planetas\footnote{Urbain Le Verrier en 1846 predijo la existencia Neptuno debido a unas inconsistencias observadas en la órbita de Urano, usando las leyes de Kepler y Newton. Años más tarde, tal vez sugestionado por su descubrimiento anterior, Le Verrier interpretó que la anomalía en la órbita de Mercurio era debido a un planeta no descubierto que llamo Vulcano. Esto se vio reforzado por un astrónomo que creyó haberlo visto en su telescopio cuando en realidad solo vió una mancha en el lente.}.
De esta manera, a pesar que la teoría gravitacional de Newton entra en conflicto con el principio de equivalencia y las leyes de conservación del momento, la validez esencial de la teoría fue establecida y aceptada durante muchos años. Antes que pensar que la teoría está mal o es incorrecta, porque además, qué está bien o qué mal, está sujeto al contexto y a los conocimientos presentes en un determinado momento de la historia. La verdad no es un absoluto, si no, un acuerdo o consenso entre las personas y esto conlleva a muchas limitaciones. En otras palabras la teoría de Newton es un modelo que sirve para describir gran cantidad de fenómenos de la naturaleza y de forma perfectamente válida, no obstante es plausible que la dicha teoría esté incompleta y necesite ser más desarrollada. En particular, debe ser reformulada de manera tal que cumpla con el principio de equivalencia y satisfaga las leyes de conservación del momento.

\newthought{El objetivo de este capítulo} es construir las ecuaciones tensoriales que describan de una vez por todas los campos gravitatorios. Para ello, comencemos pensando en la \textbf{Ley de Gauss} para el Electromagnetismo, que conecta el campo eléctrico $\mathbf{E}$ con la densidad de carga $\rho$:
    $$\nabla \cdot \mathbf{E}= 4\pi k \rho$$
Cuando desarrolló su Teoría de Gravitación Universal, Newton no contaba con el concepto de campo, pero hoy en día podemos construir un análogo a la Ley de Gauss, para el campo gravitatorio $\mathbf{g}=-\nabla \Phi$ (donde $\Phi$ es el \textbf{potencial gravitatorio}), y una densidad de masa $\rho$:
    $$-\nabla \cdot \mathbf{g}= 4\pi G \rho \hspace{0.5cm}\Longrightarrow\hspace{0.5cm} \nabla^2\Phi=4\pi G\rho$$
Esta ecuación ya tiene una característica favorable: es una \textbf{ecuación de campo local}, es decir que relaciona la derivada del campo con su fuente \textit{en el mismo punto}. Esto nos ahorra el problema de señales instantáneas que presentamos al principio, y también nos permite considerar sistemas localmente inerciales, algo a lo que ya deberíamos estar acostumbradxs. 

Nuestro objetivo será entonces generalizar esta última ecuación. En base a lo que hemos estudiado, podemos anticipar que tendrá esta forma:

\begin{figure}[h!]
    \centering
    \includegraphics[width=0.95\textwidth]{Im/eceinstein.png}
    \label{fig:sen}
\end{figure}

\newpage
\section{\huge{Tensor de Estrés-Energía}}
\textcolor{myred}{\hrule}
\begin{flushright}
\textit{Anxiety is a scalar,\\fear is a vector,\\stress is a tensor.\\\textbf{Elon Musk} (War Criminal)}
\end{flushright}
El objetivo de esta sección es generalizar la densidad de masa escalar $\rho$ a una magnitud tensorial. Es decir, vamos a trabajar con el lado derecho de la ecuación, las \textit{fuentes}.

\vspace{0.5cm}

Primero podríamos preguntarnos, ¿en la ecuación newtoniana, la densidad $\rho$ representa la masa de un objeto, o es en realidad su \textbf{energía relativista}? En el límite newtoniano, estas dos magnitudes son equivalentes. Se puede demostrar (algo que no haremos aquí), que si la fuente del campo fuera la masa, podríamos crear energía \textit{de la nada}, mientras que 
si consideramos la energía como fuente de la gravedad, esto no sucede.

\vspace{0.25cm}

Lo siguiente sería pensar si $\rho$ pudiera ser una de las entradas de un tensor, de la misma manera que la densidad de carga $\rho$ resultó ser la componente temporal $J^{0}$ de el cuadrivector densidad de corriente $\mathbf{J}$. A este último resultado (propio de la Relatividad Especial), se llega mediante la \textbf{conservación de la carga}, así que podríamos esperar llegar a algo similar aquí apelando a la \textbf{conservación de la energía}.

La construcción de este tensor no es nada trivial, y puede visualizarse mejor mediante ejemplos: la energía-momento del \textbf{polvo} (partícula que se mueven estando en reposo unas respecto de las otras), o la energía-momento de un \textbf{fluido perfecto} (un conjunto de partículas que se mueven aleatoriamente pero no interactúan entre sí, por ejemplo un gas ideal). Vamos a pasar por alto estos ejemplos y caer directamente en el tensor, explicando sus componentes y sus propiedades:
\begin{remarkbox}{Tensor de Estrés-Energía o Energía-Momento}
    \centering
    \includegraphics[width=0.85\textwidth]{Im/tensor.png}
\end{remarkbox}

De este tensor vamos a enunciar las siguientes propiedades:
\begin{itemize}
    \item Es \textbf{simétrico}, es decir que:
    $$T^{\mu\nu}=T^{\nu\mu}$$
    \item La \textbf{conservación de energía y flujo de momento} nos garantizan que:
    $$\nabla_{\nu}T^{\mu\nu}=0$$
\end{itemize}

Como vamos a estudiar la ecuación de campo de un agujero negro con carga eléctrica (Reissner-Nordström), vamos a buscar el tensor de estrés energía para un campo electromagnético.

\begin{remarkbox}{Tensor de Estrés-Energía Electromagnético}

    Recordemos primero el \textbf{tensor de campo electromagnético} $F^{\mu\nu}$:
    \begin{equation}
    F_{\alpha\beta}=
    \begin{pmatrix}
    0 & E_1/c & E_2/c & E_3/c\\
    -E_1/c & 0 & -B_3 & B_2\\
    -E_2/c & B_3 & 0 & -B_1\\
    -E_3/c & -B_2 & B_1 & 0
    \end{pmatrix}
    \label{tensorcampoelectromagnetico}
    \end{equation}

    Sabemos además que, para un campo EM, la \textbf{densidad de energía} está dada por:
    $$\rho_E=\frac{1}{2\mu_0}\left(E^2+B^2\right)$$
    
    Como los campos EM almacenan energía, tienen que tener un tensor de stress-energía asociado. Los únicos elementos que intervienen aquí son los campos electromagnéticos y el espacio-tiempo, por lo que $T^{\mu\nu}$ (simétrico) deberá ser una combinación de $F^{\mu\nu}$ (antisimétrico) y $g_{\mu\nu}$ (simétrico). Tomando esto en cuenta, y además pidiendo que la entrada $tt$ corresponda a la densidad de energía $\rho_E$ de más arriba, podemos llegar al siguiente tensor:
    \begin{equation}
    T^{\mu\nu}=\frac{1}{\mu_0}\left(\frac{1}{4}g^{\mu\nu}F^{\sigma\gamma}F_{\sigma\gamma}-F^{\mu\alpha}g_{\alpha\beta}F^{\nu\beta}\right)
    \label{tensorestresenergiafunciondeF}
    \end{equation}
    
    
    Una característica destacable de este tensor es que es \textbf{traceless} (traza nula). Es decir que, si bajamos un índice y contraemos, podemos demostrar que:
    
    \begin{equation}
        T\equiv T^{\alpha}_{\alpha}=0
        \label{traceless}
    \end{equation}
\end{remarkbox}

\newpage

\section{\huge{Ecuaciones de Campo de Einstein}}

\textcolor{myred}{\hrule}

\newthought{Ya obtuvimos} lo que va al lado derecho de las ecuaciones de campo. Lo que nos queda por hacer es construir el lado izquierdo de la misma. Evidentemente, el dado izquierdo también deberá:

\begin{itemize}
    \item Ser un \textbf{tensor contravariante de segundo rango}, al que llamaremos $G^{\mu\nu}$, tal que:
    
    $$G^{\mu\nu}=k T^{\mu\nu}$$
    
    ($k$ es una constante de proporcionalidad a determinar).
    
    \item Ser \textbf{simétrico}, es decir que $G^{\mu\nu}=G^{\nu\mu}$.
    \item Tener \textbf{gradiente absoluto nulo}, es decir $\nabla_{\mu}G^{\mu\nu}=0.$
    \item Representar la \textbf{curvatura} del espacio-tiempo, es decir que deberá \textbf{contener derivadas segundas del tensor métrico} $g_{\mu\nu}$.
    \item Reducirse a la fórmula newtoniana $\nabla^2\Phi=4\pi G \rho$ en el límite no-relativista.
\end{itemize}

Una primer propuesta que podríamos hacer es utilizar el \textbf{tensor de Ricci} (que, si recordamos, era una contracción del tensor de Riemann, el cual contiene toda la información necesaria para describir la curvatura del espacio-tiempo). Si bien este tensor es simétrico, el problema yace en que, en general, $\nabla_\mu R^{\mu\nu}\neq0$, por lo que tendremos que proponer algo distinto.

\vspace{0.5cm}

¿Qué otras cosas podemos agregar? Los únicos elementos razonables podrían ser otras contracciones del tensor de Riemann (por ejemplo, el escalar de curvatura $R$), y el propio tensor métrico $g^{\mu\nu}$. La forma más general de combinar esto (respetando el requisito de simetría) es la siguiente:

$$G^{\mu\nu}=R^{\mu\nu}+b g^{\mu\nu} R+\Lambda g^{\mu\nu}$$

Donde $b$ y $\Lambda$ son constantes (escalares).

Si pedimos que, además, $\nabla_{\nu}G^{\mu\nu}=0$, tenemos que:

\begin{equation}
\begin{split}
\nabla_{\nu}G^{\mu\nu}&=\nabla_{\nu}\left(R^{\mu\nu}+b g^{\mu\nu} R+\Lambda g^{\mu\nu}\right)=\\
&=\nabla_{\nu}\left(R^{\mu\nu}+b g^{\mu\nu} R\right)=0
\end{split}
\end{equation}

\begin{marginfigure}
\begin{remarkbox}{Identidad de Bianchi}
$$\nabla_\sigma R_{\alpha\beta\mu\nu}+\nabla_\nu R_{\alpha\beta\sigma\mu}+$$ $$\nabla_\mu R_{\alpha\beta\nu\sigma}=0$$
\end{remarkbox}
\end{marginfigure}

Si utilizamos la \textbf{identidad de Bianchi}, podemos demostrar que es necesario que $b=-\frac{1}{2}$. La ecuación toma entonces la forma:

\begin{equation}
    R^{\mu\nu}-\frac{1}{2} g^{\mu\nu} R+\Lambda g^{\mu\nu}=k T^{\mu\nu}
\end{equation}

Podemos utilizar ahora el requisito de que la ecuación se reduzca a la newtoniana en el límite no-relativista para determinar el valor de la constante $k$. Esto consiste en considerar el \textbf{límite de campo débil}\cite[][p.255]{moore}, e implementar las desviaciones geodésicas. Nos saltearemos esta deducción\cite[][p.246]{moore} para concluir que es necesario que $k=8\pi G$. Las ecuaciones de campo son entonces:

\begin{marginfigure}
\begin{remarkbox}{Weak-Field Limit}
El \textbf{Límite de Campo Débil} consiste en proponer un tensor métrico que difiere muy levemente del $\eta_{\mu\nu}$ del espacio de Minkowski. Es posible demostrar, mediante un análisis perturbativo, que en dicho límite la entrada $g_{00}$ del tensor métrico tiene la forma:

\begin{equation}
    g_{00}=1-\frac{2GM}{c^2 r}
    \label{g00}
\end{equation}

Que está íntimamente relacionada con el potencial gravitatorio newtoniano.
\end{remarkbox}
\end{marginfigure}

\begin{remarkbox}{Ecuaciones de Campo de Einstein}
\begin{equation}
    R^{\mu\nu}-\frac{1}{2} g^{\mu\nu} R+\Lambda g^{\mu\nu}=8 \pi G T^{\mu\nu}
\end{equation}
\end{remarkbox}

La constante $\Lambda$ se llama \textbf{constante cosmológica}, y en los análisis posteriores podemos considerarla igual a cero.

Multiplicando por el tensor métrico y contrayendo un par de índices, podemos llegar a una forma alternativa para las ecuaciones:

\begin{remarkbox}{Ecuaciones de Campo de Einstein - Forma Alternativa}
\begin{equation}
    R^{\mu\nu}=8 \pi G\left(T^{\mu\nu}-\frac{1}{2} g^{\mu\nu} T\right)+\Lambda g^{\mu\nu}
    \label{campoeinsteincontravariante}
\end{equation}
\end{remarkbox}

\newpage

\section{\huge{Geodésicas}}
\textcolor{myred}{\hrule}
\begin{flushright}
\textit{It took the light,\\ absolutely forever to get to your eyes.\\\textbf{Alex Turner}}
\end{flushright}

\newthought{Ya sabemos a la perfección cómo es que la energía modifica la geometría del espacio-tiempo}. Lo que nos queda por explicar ahora es cómo un objeto se mueve en este espacio-tiempo curvado, para poder hallar ecuaciones diferenciales de movimiento. 

Para esto, vamos a definir una \textbf{geodésica} como la \textbf{curva entre dos eventos con el mayor tiempo propio}. La hipótesis geodésica dice que, en los espacio-tiempos con cualquier tipo de geometría (incluso curvados), los cuerpos libres de fuerzas se mueven en geodésicas. Es decir que, al viajar de un punto $A$ a un punto $B$ en el espacio-tiempo, de todas las posibles trayectorias los cuerpos 'eligen' la que hace que sus relojes midan más tiempo propio.

Veamos qué se puede hacer con esta hipótesis.

\subsection*{\textbf{Geodésicas Timelike}}

\begin{marginfigure}
\captionsetup{type=figure}
    \centering
    \includegraphics[width=1.3\textwidth]{Im/geod.png}
    \caption{Podemos describir la curva (worldline) entre dos eventos $A$ y $B$ etiquetando todos los eventos entre ellos mediante un parámetro $\sigma$, luego especificando las coordenadas espaciales y temporales de dichos eventos como función de dicho parámetro.}
    \label{fig:sen}
\end{marginfigure}

Tomemos dos eventos $A$ y $B$ que tienen una separación de tipo \textit{timelike} (es decir con $ds^2>0$, el movimiento de las partículas con masa, que viajan a velocidades inferiores a la de la luz). Pensemos en las posibles trayectorias que conectan los dos eventos, y hagámoslas variar con un parámetro $\sigma$, que va de $0$ a $1$. Es decir, vamos a especificar $x^{\mu}(\sigma)$ a lo largo de la trayectoria. El tiempo propio a lo largo de esta trayectoria va a estar dado por:

\begin{equation}
    \tau_{AB}=\int \sqrt{-ds^2}=\int_0^1\sqrt{-g_{\mu\nu}(x^{\alpha}(\sigma))\frac{dx^\mu}{d\sigma}\frac{dx^\nu}{d\sigma}}d\sigma
\end{equation}

Hallar la curva que maximice el valor de $\tau_{AB}$ es muy similar a cuando, en Mecánica Clásica, buscamos la trayectoria $q_i(t)$ que \textbf{minimice la acción $S$}, la cual estaba definida como:

$$S\equiv \int_{t_a}^{t_b}L(q_i,\Dot{q}_i)dt$$

Donde $L$ es el \textbf{lagrangiano} del sistema, una función de las coordenadas generalizadas y las velocidades generalizadas. A partir de este principio variacional, llegábamos a las \textbf{Ecuaciones de Euler-Lagrange}:

\begin{equation}
    \frac{d }{dt}\left(\frac{\partial L}{\partial \dot{q}_i}\right)-\frac{\partial L}{\partial q_i}=0
\end{equation}

La situación aquí es completamente análoga. Podemos armar un lagrangiano que depende de las posiciones $x^{\mu}$ y las 'velocidades' $\dot{x}^\mu \equiv d x^\mu / d\sigma$:

\begin{remarkbox}{Lagrangiano}
\begin{equation}
    L(x^{\alpha},\dot{x}^{\alpha})\equiv\sqrt{-g_{\mu\nu}(x^{\alpha})\dot{x}^{\mu}\dot{x}^{\nu}}
\end{equation}
\end{remarkbox}

Que, de igual manera, nos conducen a las ecuaciones:

\begin{remarkbox}{Ecuaciones de Euler-Lagrange}
\begin{equation}
    \frac{d }{d\sigma}\left(\frac{\partial L}{\partial \dot{x}^{\alpha}}\right)-\frac{\partial L}{\partial x^{\alpha}}=0
\end{equation}
\end{remarkbox}

Se puede proceder reemplazando el lagrangiano y derivando, y luego de un buen trabajo algebraico, llegamos a la siguiente ecuación:

\begin{remarkbox}{Ecuación geodésica}
\begin{equation}
\frac{d}{d\tau}\left(g_{\alpha\beta}\frac{dx^{\beta}}{d\tau}\right)-\frac{1}{2}\partial_\alpha g_{\mu\nu}\frac{dx^{\mu}}{d\tau}\frac{dx^{\nu}}{d\tau}=0
\end{equation}
\end{remarkbox}

Esta ecuación es muy conveniente, ya que utiliza el tensor métrico de forma explícita, en lugar de tenerlo 'oculto' en el lagrangiano. También nos hemos independizado del parámetro $\sigma$, aunque a veces nos será conveniente volver a dicho parámetro arbitrario, como es en el caso de las geodésicas \textit{lightlike}, es decir la trayectoria de fotones. Para cualquier trayectoria, el $\tau_{AB}$ de la luz es siempre cero (podríamos decir que \textit{un reloj moviéndose a la velocidad de la luz no avanza}), por lo que $\tau$ deja de ser un buen parámetro.

También es posible utilizar los \textbf{símbolos de Christoffel} para reescribir la ecuación geodésica. Esta se lee:

\begin{remarkbox}{Ecuación geodésica - Forma Alternativa}
\begin{equation}
\frac{d^2x^{\mu}}{d\tau^2}+\Gamma^{\mu}_{\alpha\beta}\frac{dx^{\alpha}}{d\tau}\frac{dx^{\beta}}{d\tau}=0
\label{geodesicx}
\end{equation}
\end{remarkbox}

\begin{figure}
    \centering
    \includegraphics[width=0.6\textwidth]{Im/sonic.jpg}
    \caption{El único meme que pusimos en este trabajo.}
    \label{fig:my_label}
\end{figure}






%Einstein Field equation
%\chapter{\textcolor{myred}{La métrica de Reissner-Nordström}}

%Principio de equivalencia (fuerte): En un laboratorio en caída libre (no rotante) que ocupa una pequeña región del espacio tiempo, las leyes de la física son las correspondientes a la relatividad especial

\section{\huge{Resolviendo la ecuación de Einstein}}

\textcolor{myred}{\hrule}

\newthought{En los capítulos} anteriores hemos definido una buena cantidad de conceptos y herramientas correspondientes a la teoría de la Relatividad General. En este capítulo, buscaremos una solución a la ecuación de campo de Einstein-Maxwell para un cuerpo\footnote{Notesé como nos referimos a un cuerpo con ciertas características, no a un agujero negro en particular.} puntual con carga. 

\newthought{El primer paso} para resolver la ecuación de campo de Einstein es definir un buen sistema de coordenadas. Hemos visto en los capítulos anteriores que la geometría del espacio-tiempo alrededor de una distribución de masa y energía está determinada por dicha ecuación. No obstante, tenemos libertad para elegir las coordenadas que utilicemos para describir dicha geometría. Como hacemos siempre en Física, utilizaremos un sistema de coordenadas que explote las simetrías del problema que estemos resolviendo. Luego teniendo en cuenta, estas simetrías podremos proponer una métrica que contenga la menor cantidad de componentes incógnitas posibles.
\begin{marginfigure}
\begin{extrabox}{}
Los pasos para resolver la ecuación se pueden condensar en:
\begin{itemize}
    \item Usar la simetría del problema para definir un sistema de coordenadas lo más completo posible.
    \item Proponer una métrica con la menor cantidad de coeficientes indeterminados posibles.
    \item Sustituir la métrica de prueba en la ecuación de Einstein.
    \item Resolver el sistema de ecuaciones diferenciales para las componentes incógnitas.
\end{itemize}
\end{extrabox}
\end{marginfigure}
Una vez que tengamos esta métrica de prueba la sustituimos en la ecuación de Einstein y obtenemos un sistema de ecuaciones diferenciales acopladas, el cual, si somos capaces de resolverlo, habremos encontrado un sistema de coordenadas que describe la geometría buscada, es decir obtuvimos la métrica para del  espacio-tiempo para el sistema estudiado.

\section{\huge{La forma general de una métrica esféricamente simétrica}}

\textcolor{myred}{\hrule}

\subsection*{\textbf{Consideraciones sobre la forma general de la métrica}}

\newthought{Para poder resolver} las ecuaciones, tendremos que asumir ciertas características de la solución:

\vspace{0.75cm}

\begin{remarkbox}{Consideraciones sobre la métrica de Reissner-Nordström}
\begin{itemize}
    \item Consideramos el espacio tiempo alrededor de una fuente esféricamente simétrica.
    \item El espacio es vacío excepto por la presencia de campos electromagnéticos.
    \item Cuando la carga del cuerpo tiende a 0 $(Q \to 0)$ la métrica debe ser la de Schwarzchild.
    \item Cuando la distancia al objeto tiende a infinito $(r \to \infty)$ la métrica debe acercarse a la de Minkowski.
\end{itemize}
\end{remarkbox}

\subsection*{\textbf{Elección de las coordenadas espaciales}}

Simetría esférica significa que podemos definir un conjunto anidado de superficies bidimensionales concéntricas en el espacio-tiempo alrededor de la fuente cuya geometría intrínseca es la misma que la de una esfera ordinaria en dos dimensiones. Si definimos las coordenadas angulares $\theta$ (ángulo polar) y $\phi$ (ángulo azimutal) en la forma usual para cada superficie, y definimos $r$ como la circunferencia de tal esfera dividida por $2\pi$. Luego la métrica para cada superficie esférica es 
\begin{equation}
    ds^2 = r^2(d\theta^2+\sin^2{\theta}d\phi^2)
    \label{metricaRN1}
\end{equation}

Esta ''parte'' de la métrica solo aplica para una de las esferas anidadas ($r$ constante), es decir tenemos las componentes del tensor métrico que tienen que ver con las coordenadas angulares. Hasta ahora no hay nada que nos prohiba darle a cada esfera su propio set de coordenadas angulares. Pero podemos alinear los sistemas coordenados de todas las esferas concéntricas, pidiendo que la linea curva definida por $\theta = cte$ y $\phi=cte$ sean perpendiculares a cada esfera. Esto significa que los vectores $\bm{e}_\theta$ y $\bm{e}_\phi$ sean perpendiculares a $\bm{e}_r$. Recordando \ref{tensormetricobases} lo anterior nos dice que
\begin{equation*}
    g_{r\theta} = \bm{e}_r  \cdot \bm{e}_\theta = 0 \hspace{1cm} g_{r\phi} = \bm{e}_r  \cdot \bm{e}_\phi =0
\end{equation*}
Entonces, tenemos las componentes del tensor métrico que tienen que ver con $r$, donde $g_{rr}$ es incógnita por ahora y podemos escribir la parte puramente espacial de la métrica
\begin{equation}
    ds^2 = g_{rr}dr^2 + r^2(d\theta^2+\sin^2{\theta}d\phi^2)
    \label{metricaRN2}
\end{equation}

Ahora, supongamos que tenemos un término de la métrica que sea de la forma $g_{t\phi}dtd\phi$, esto implicaría que la geometría del espacio tiempo trata a los desplazamientos dondo $d\phi > 0$ distinto a los desplazamientos donde $d\phi <0$. Esto daría una dirección preferencial al movimiento contrario a la suposición de simetría esférica\footnote{La métrica de Kerr para agujeros negros rotantes, donde no hay simetría esférica esta componente del tensor no es nula, lo cual complica considerablemente las cosas.}. Entonces, la simetría esférica del sistema nos permite elegir usar coordenadas de modo tal que $g_{t\phi}=0$. Por un argumento, similar $g_{t\theta}=0$. Finalmente la métrica que tiene en cuenta el espacio y el tiempo tiene la forma 
\begin{equation}
    ds^2 = g_{tt}dt^2 + 2g_{rt}drdt +g_{rr}dr^2 + r^2(d\theta^2+\sin^2{\theta}d\phi^2)
\end{equation}
donde escribimos $2g_{rt}drdt$ por la simetría del tensor métrico. 

\subsection*{\textbf{Elección de la coordenada temporal}}

Hasta ahora la coordenada temporal es completamente arbitraria. Podemos usar nuestra libertad en la elección de las coordenadas para definir $g_{tt}$ y $g_{rt}$, pero teniendo cuidado que hay ciertas restricciones. Primero, $g_{tt} \neq 0$ o si no, no tendríamos espacio-\textit{tiempo}. Segundo, debe ser negativo cuando las otras componentes de la métrica son positivos y $g_{tr}=0$ (para que exista el caso lightlike $ds^2=0$). Supongamos que realizamos alguna elección arbitraria de las coordenadas temporales que definen $g_{tt}$ y utilizamos las ecuaciones de Einstein para encontrar $g_{rt}$. Si realizamos una transformación de coordenadas de la forma 
\begin{equation*}
    t' = t + f(r,t)
\end{equation*}
se puede demostrar que una elección apropiada de dicha función setea $g'_{rt}=0$ en nuestro sistema de coordenadas transformado\footnote{El ' significa en el sistema transformado, no derivada con respecto a algo.}. Pero si podemos elegir una coordenada temporal de modo tal que se cumpla $g'_{rt}=0$, por la arbritrariedad en la elección podemos elegir directamente $g_{rt}=0$ antes de resolver la ecuación de Einstein.\footnote{Eliminar elementos de la diagonal representa una gran ventaja a la hora de realizar los cálculos.}.

Finalmente, una forma general para la métrica esféricamente simétrica es
\begin{remarkbox}{Forma general para una métrica esféricamente simétrica }
\begin{equation*}
    ds^2 = A(r,t)dt^2 - B(r,t)dr^2 - r^2d\theta^2 -r^2\sin^2{\theta}d\phi^2
\end{equation*}
\end{remarkbox}

donde elegimos $\phi$ y $\theta$ como las coordenadas angulares azimutal y polares de siempre, $r$ como la coordenada radial circunferencial, y $t$ de modo tal que $g_{rt}$.

\section{\huge{Reemplazando la forma general de la métrica en la ecuación de Einstein}}

\newthought{Siguiendo} la 'receta' que presentamos antes, ahora debemos utilizar la forma de la métrica que hallamos y reemplazarla directamente en la ecuación de campo de Einstein que recordemos está dada por (\ref{campoeinsteincontravariante}), tomando la constante cosmológica como $0$ tenemos\footnote{Hemos agregado un término $c^{-4}$ que antes no estaba por simplicidad de las cuentas pero ahora nos será útil trabajar en el Sistema Internacional de Unidades (SI).}
\begin{equation*}
R^{\mu\nu}= \frac{8 \pi G}{c^4}\left(T^{\mu\nu}-\frac{1}{2} g^{\mu\nu} T\right)
\end{equation*}
Además, recordando que estamos trabajando en el vacío (solo con presencia de campos electromagnéticos), el tensor de energía-momento está dado por (\ref{tensorestresenergiafunciondeF})
\begin{equation*}
T^{\mu\nu}=\frac{1}{4\pi k}\left(\frac{1}{4}g^{\mu\nu}F^{\sigma\gamma}F_{\sigma\gamma}-F^{\mu\alpha}g_{\alpha\beta}F^{\nu\beta}\right)
\end{equation*}
y para este tensor calculamos que su traza era nula (ec. \ref{traceless}), por lo que la ecuación de campo de Einstein que usaremos es
\begin{equation}
R^{\mu\nu}=\frac{8 \pi G}{c^4}T^{\mu\nu}
\label{riccicosmonula}
\end{equation}
\subsection*{\textbf{Bajar los índices 'upstairs' a 'downstairs'}}

En esta sección expresaremos el tensor de estrés-energía (ec. \ref{tensorestresenergiafunciondeF} y el tensor de Ricci (ec. \ref{riccicosmonula}) de forma covariante (con los índices abajo \textit{downstairs}), dado que nos será útil más adelante. 

Para bajar dichos índices multiplicamos a ambos lados por el tensor métrico\begin{marginfigure}
\begin{extrabox}{}
Para bajar o subir índices una regla que funciona es pensar que el segundo índice del tensor métrico índica qué índice del tensor sube o baja (es el que está repetido) y el primer índice del tensor métrico indica por qué índice se remplaza. Por ejemplo:
\begin{equation}
    g^{\alpha\gamma} R_{\alpha\beta\gamma\sigma} = {{R_{\alpha\beta}}^\alpha}_\sigma
\end{equation}
\end{extrabox}
\end{marginfigure} de forma conveniente entonces para el tensor de Ricci por ejemplo:
\begin{equation}
\begin{split}
g_{\alpha\mu}g_{\beta\nu}R^{\mu\nu} &= \frac{8 \pi G}{c^4}g_{\alpha\mu}g{\beta\nu}T^{\mu\nu} \\    
g_{\alpha\mu}R^{\mu}_\beta &= \frac{8 \pi G}{c^4}g_{\alpha\mu}T^{\mu}_\beta \\    
R_{\alpha\beta} &= \frac{8 \pi G}{c^4}T_{\alpha\beta}   
\end{split}
\label{riccidownstairs}
\end{equation}
De manera similar podemos bajar los indices del tensor de estrés-energía
\begin{equation}
T_{\alpha\beta}=\frac{1}{\mu_0}\left(\frac{1}{4}g_{\alpha\beta}F_{\mu\nu}F^{\mu\nu}-g_{\beta\nu}F_{\alpha\mu}F^{\nu\mu}\right)
\label{tensorestresdownstairs}
\end{equation}

\subsection*{\textbf{Cálculo del tensor de Ricci}}

Comencemos por el lado izquierdo de la ecuación de Einstein en forma covariante (\ref{riccidownstairs}), es decir por el cálculo del tensor de Ricci. Para calcular las componentes de dicho tensor debemos primero calcular los símbolos de Christoffel a partir de la ecuación (\ref{simboloChristoffel}) 
\begin{equation*}
\Gamma^{\alpha}_{\mu \nu}=\frac{1}{2}g^{\alpha\sigma}\left[\partial_{\mu}g_{\nu\sigma}+\partial_{\nu}g_{\sigma \mu}-\partial_{\sigma}g_{\mu\nu}\right]
\end{equation*}

Por lo que necesitamos el tensor métrico en su forma covariante y contravariante. Hasta ahora, a partir de la métrica que tenemos $g_{\mu\nu}$ y $g^{\mu\nu}$ son\footnote{Ya vimos que en la representación matricial son matrices inversas.} 
\begin{equation}
g_{\mu\nu} = \begin{pmatrix}
Ac^2 & 0 & 0 & 0 \\
0 & -B & 0 & 0 \\
0 & 0 & -r^2 & 0 \\
0 & 0 & 0 & -r^2\sen^2{\theta}
\end{pmatrix}
\hspace{0.2cm}
g^{\mu\nu} = \begin{pmatrix}
\frac{1}{Ac^2} & 0 & 0 & 0 \\
0 & \frac{-1}{B}  & 0 & 0 \\
0 & 0 & \frac{-1}{r^2}  & 0 \\
0 & 0 & 0 & \frac{-1}{r^2\sen^2{\theta}} 
\end{pmatrix}
\label{tensoresmetricos}
\end{equation}

Veamos que para calcular los símbolos de Christoffel solo debemos calcular las derivadas parciales de las entradas del tensor métrico y ser muy prolijos con los índices. Esta tarea no es para nada difícil, pero sí muy larga y tediosa. A modo de ejemplo, mostramos en el apéndice ... el cálculo del símbolo ..... Esta tarea por suerte puede ser simplificada mediante métodos computacionales\cite[][p. 26]{Jhonny} y los símbolos de Christoffel no nulos resultan\footnote{Existen varios softwares que permiten el cálculo de simbolos de Christoffel, tensores de Ricci, de Riemann. Maple 17 es un software que permite realizar este tipo de cálculo para entradas incógnitas como los A y B que tenemos, lamentablemente es privado. Einstein.py es un paquete de Python de carácter libre que también permite realizar este tipo de cálculos, pero (hasta donde sabemos) solo para métricas ya dadas. Es decir, no podemos realizar derivadas simbólicas para los $A$ y $B$ que necesitamos.} 

\begin{equation}
\begin{split}
&\Gamma^{0}_{00} = \frac{\dot{A}}{2Ac} \hspace{1cm} \Gamma^{1}_{01} = \Gamma^{1}_{10} = \frac{\dot{B}}{2Bc} \\
&\Gamma^{0}_{11} = \frac{\dot{B}}{2Ac} \hspace{1cm} \Gamma^{0}_{01} = \Gamma^{0}_{10} = \frac{A'}{2A} \\
&\Gamma^{1}_{00} = \frac{A'}{2B} \hspace{1cm} ~~\Gamma^{2}_{12} = \Gamma^{2}_{21} = \frac{1}{r} \\
&\Gamma^{1}_{11} = \frac{B'}{2B} \hspace{1cm} ~~\Gamma^{3}_{13} = \Gamma^{3}_{31} = \frac{1}{r} \\
&\Gamma^{1}_{22} = -\frac{r}{B} \hspace{1cm} ~\Gamma^{3}_{23} = \Gamma^{3}_{32} = \cot{\theta} \\
&\Gamma^{1}_{33} = -\frac{r\sen^2{\theta}}{B} \hspace{0.5cm} \Gamma^{2}_{33} = -\sen{\theta}\cos{\theta} \\
\end{split}
\end{equation}

donde las primas son derivaciones con respecto a $r$ y los puntos con respecto a $t$.

Luego a partir de la definición del tensor de Riemann (ec. \ref{riemann}) 
\begin{equation*}
R^\mu_{\alpha \nu \beta}\equiv\partial_{\nu}\Gamma^{\mu}_{\alpha \beta}-\partial_{\beta}\Gamma^{\mu}_{\alpha\nu}+\Gamma^{\mu}_{\nu\gamma}\Gamma^{\gamma}_{\alpha\beta}-\Gamma^{\mu}_{\beta\gamma}\Gamma^{\gamma}_{\alpha\nu}
\end{equation*}
y del tensor de Ricci (ec. \ref{ricci}) 
\begin{equation*}
R_{\alpha\beta} \equiv R^\mu_{\alpha \mu \beta}
\end{equation*}

Lo único que tenemos que hacer para obtener el tensor de Ricci en función de los símbolos de Christoffel es intercambiar los $\nu$ por $\mu$ y obtenemos
\begin{equation}
R^\mu_{\alpha \mu \beta}= R_{\alpha\beta}= \partial_{\mu}\Gamma^{\mu}_{\alpha \beta}-\partial_{\beta}\Gamma^{\mu}_{\alpha\mu}+\Gamma^{\mu}_{\mu\gamma}\Gamma^{\gamma}_{\alpha\beta}-\Gamma^{\mu}_{\beta\gamma}\Gamma^{\gamma}_{\alpha\mu}    
\end{equation}
y entonces los simbolos de Ricci son:
\begin{equation}
\begin{split}
R_{00} &= -\frac{A'}{4B}\biggr(\frac{A'}{A}+\frac{B'}{B}\biggr)+\frac{A''}{2B}+\frac{A'}{Br}-\frac{\Ddot{B}}{2Bc^2}+\frac{\dot{B}}{4Bc^2}\biggr(\frac{\dot{A}}{A}-\frac{\dot{B}}{B}\biggr) \\
R_{11} &= \frac{A'}{4A}\biggr(\frac{A'}{A}+\frac{B'}{B}\biggr)-\frac{A''}{2A}+\frac{B'}{Br}-\frac{\Ddot{B}}{2Ac^2}-\frac{\dot{B}}{4Ac^2}\biggr(\frac{\dot{A}}{A}-\frac{\dot{B}}{B}\biggr) \\
R_{22} &= -\frac{r}{2B}\biggr(\frac{A'}{A}-\frac{B'}{B}\biggr)-\frac{1}{B}+ 1 \\
R_{33} &= \biggr[-\frac{r}{2B}\biggr(\frac{A'}{A}-\frac{B'}{B}\biggr)-\frac{1}{B}+ 1 \biggr]\sin^2{\theta} = R_{22}\sin^2{\theta}\\
R_{01} &= R_{10} = \frac{\dot{B}}{Brc} \\
\end{split}
\end{equation}

\newthought{Hasta aquí} es lo máximo que podemos generalizar el campo gravitacional esféricamente simétrico. Para determinar $A$ y $B$ necesitaremos resolver el lado derecho de la ecuación de Einstein utilizando el tensor de energía-momento y luego comparar con el lado izquierdo. 

\subsection*{\textbf{Cálculo del Tensor de Estrés-Energía}}
En nuestro caso, solo hay campos eléctricos en el espacio, y sabemos de la simetría esférica del problema que el campo solo puede tener componente radial y que dicha componente solo no puede depender de $\phi$ o $\theta$, entonces
\begin{equation}
    E_1=E_r(t,r)=cF_{01}=-cF_{10}
\end{equation}
El tensor de campo electromagnético dado por (\ref{tensorcampoelectromagnetico}) queda como
\begin{equation}
    F_{\alpha\beta}=
    \begin{pmatrix}
    0 & E_r/c & 0 & 0\\
    -E_r/c & 0 & 0 & 0\\
    0 & 0 & 0 & 0\\
    0 & 0 & 0 & 0
    \end{pmatrix}
\end{equation}

Y ahora podemos usar (\ref{tensorestresdownstairs}) para calcular las componentes del tensor de estrés-energía. El primer término es
\begin{equation*}
\begin{split}
\frac{1}{4}g_{\alpha\beta}F_{\mu\nu}F^{\mu\nu} &= \frac{1}{4}g_{\alpha\beta}(F_{\mu 0}F^{\mu0} + F_{\mu 1}F^{\mu1})= \frac{1}{4}g_{\alpha\beta}(F_{00}F^{00} + F_{10}F^{10} + F_{0 1}F^{01}+ F_{11}F^{11}) \\
&= \frac{1}{2}g_{\alpha\beta}F_{0 1}F^{01}
\end{split}
\end{equation*}
donde sumamos sobre los índices repetidos, usamos que el tensor $F_{\alpha\beta}$ solo tiene componentes no nulas $F_{10}$ y $F_{01}$ y que es antisimétrico. El segundo término nos da
\begin{equation}
g_{\beta\nu}F_{\alpha\mu}F^{\nu\mu} = g_{\beta\nu}F_{\alpha 0}F^{\nu 0} + g_{\beta\nu}F_{\alpha 1}F^{\nu 1} = g_{\beta 1}F_{\alpha 0}F^{10} + g_{\beta 0}F_{\alpha 1}F^{01}
\end{equation}
y finalmente llegamos a una expresión para el tensor
\begin{equation}
T_{\alpha\beta} = \frac{1}{\mu_0}\biggr(\frac{1}{2}g_{\alpha\beta}F_{0 1}F^{01} - g_{\beta 1}F_{\alpha 0}F^{10} - g_{\beta 0}F_{\alpha 1}F^{01}\biggr)
\end{equation}
A partir de esto obtener las componentes del tensor fácil, simplemente reemplazamos $\alpha$ y $\beta$ por todas las combinaciones posibles entre $0$ y $3$ y reemplazamos las entradas del tensor $g_{\alpha\beta}$ según (\ref{tensoresmetricos}) y obtenemos que los únicos 4 términos no nulos 
\begin{equation}
\begin{split}
T_{00} &= \frac{1}{\mu_0}\biggr(\frac{1}{2}g_{00}F_{01}F^{01} - g_{00}F_{01}F^{01}\biggr)= -\frac{1}{2\mu_0}AF_{01}F^{01}\\
T_{11} &=\frac{1}{2\mu_0}B F_{01}F^{01}\\
T_{22} &= -\frac{1}{2\mu_0}r^2 F_{01}F^{01} \\
T_{33} &= \frac{1}{2\mu_0}r^2 \sen^2{\theta} F_{01}F^{01} = T_{22}\sen^2{\theta}
\end{split}
\label{componentestensorestresenergia}
\end{equation}

Ahora que tenemos ambos lados de la ecuación de Einstein estamos en condiciones de comparar el tensor de Ricci con el de energía-momento para poder finalmente calcular $A$ y $B$ y obtener nuestra métrica.

\subsection*{\textbf{Comparación tensor de Ricci y de Estrés-Energía}}

Es fácil, ver de la ecuación de campo que si una entrada en el tensor de Ricci es nula lo será en el de estrés-energía y viceversa. Por ejemplo, la entrada $T_{01}$ es nula entonces la entrada $R_{01}$ deberá serlo y de la ecuación para $R_{01}$ podemos concluir que
\begin{equation}
    \dot{B} = 0 
\end{equation}
$B$ no depende de $t$.

Por otro lado, de (\ref{componentestensorestresenergia}) podemos ver que se cumple
\begin{equation}
    \frac{T_{00}}{A} + \frac{T_{11}}{B} = 0 
\end{equation}
Como sabemos que cada coordenada de $R_{\alpha\beta}$ es una constante multiplicada por $T_{\alpha\beta}$ (ec. \ref{riccidownstairs}) podemos escribir
\begin{equation}
\frac{R_{00}}{A} + \frac{R_{11}}{B} = \frac{1}{rB}\biggr(\frac{A'}{A}+\frac{B'}{B}\biggr) = 0 
\end{equation}
lo cual implica que 
\begin{equation}
0 = \biggr(\frac{A'}{A}+\frac{B'}{B}\biggr) \Rightarrow  \frac{1}{AB}(A'B+AB') = \frac{\partial}{\partial r} \ln(AB)
\end{equation}
entonces $AB$ debe ser constante con respecto a $r$ y lo podemos escribir como 
\begin{equation}
    AB = f(t)
\end{equation}
La relación entre $F_{01}$ downstairs y $F^{01}$ upstairs está dada por
\begin{equation}
    F_{01} = g_{00}g_{11}F^{01} = -fF^{01}
\end{equation}
donde usamos que $g_{00}=A$ y $g_{11}=-B$ y que $AB=f$ 

En este punto necesitamos las ecuaciones de Maxwell en su forma tensorial
\begin{equation}
\begin{split}
\nabla_\beta F^{\alpha\beta} &= 0 \\
\nabla^\mu F^{\alpha\beta} + \nabla^\beta F^{\mu\alpha}  +\nabla^\alpha F^{\beta\mu} &= 0
\end{split}
\end{equation}

donde $\nabla_\gamma$ es la derivada covariante (ó gradiente absoluto) que ya hemos definido en el capítulo anterior.

\begin{marginfigure}
\begin{remarkbox}{Teorema de Birkhoff}
Para demostrar que la función $A$ no depende del tiempo, es posible apelar al siguiente argumento: como sabemos que $AB=f(t)$,  que $\dot{B}=0$, es necesario que $A$ tenga la siguiente forma:

$$A(r,t)=a(r)f(t)$$

Donde $a=k/B$. El coeficiente que acompaña al término $dt^2$ en la métrica será entonces $-Kf(t)g(r)$, de donde concluimos que $Kf(t)$ tiene que ser positivo (para que la \textit{signature} de la métrica nos asegure 3 dimesiones espaciales y una temporal). Esto significa que podemos redefinir la coordenada $t$ como:

\begin{equation}
    dt_{new}=dt_{old}\sqrt{Kf(t)}
\end{equation}

Es decir que la libertad a la hora de elegir la coordenada $t$ nos permite independizarnos del tiempo en la métrica alrededor de un objeto con simetría esférica. Esto es conocido como \textbf{Teorema de Birkhoff}, y si bien aplica a el espacio vacío, también puede generalizarse para incorporar las Ecuaciones de Maxwell en el llamado \textbf{Electrovacío de Reissner-Nordström}.
\end{remarkbox}
\end{marginfigure}


Luego la primera ecuación de Maxwell se escribe como
\begin{equation}
\nabla_\beta F^{\alpha\beta} = \partial_\beta F^{\alpha\beta} + \Gamma^\alpha_{\mu\beta}F^{\mu\beta} + \Gamma^\beta_{\mu\beta}F^{\alpha\mu} = 0 
\end{equation}
Si tomamos $\alpha=1$ en la anterior tenemos 
\begin{equation}
\partial_0 F^{10} + \Gamma^1_{\mu\beta}F^{\mu\beta} + \Gamma^\beta_{\mu\beta}F^{1\mu} = 0
\end{equation}
donde el termino con la derivada $\partial_0$ es el único que sobrevive por las entradas no nulas del $F_{\alpha\beta}$. Veamos la suma implícita en el segundo término
\begin{equation}
\Gamma^1_{\mu\beta}F^{\mu\beta} = \Gamma^1_{\mu0}F^{\mu0} +\Gamma^1_{\mu1}F^{\mu1}= \Gamma^1_{00}F^{00} + \Gamma^1_{10}F^{10} + \Gamma^1_{10}F^{10} + \Gamma^1_{11}F^{11}=0
\end{equation}
Luego este término es nulo porque $\Gamma^1_{01}=\Gamma^1_{10}=F^{00}=F^{11}=0$. Y el tercer término también es nulo 
\begin{equation}
\Gamma^\beta_{\mu\beta}F^{1\mu} = F^{10}(\Gamma^0_{00}+\Gamma^1_{01}+\Gamma^2_{02}+\Gamma^3_{03})=0
\end{equation}
donde los símbolos de Christoffel en el paréntesis se anulan, entonces llegamos a 

\begin{equation}
    \partial_0 F^{10} = 0
\end{equation}
Esto implica, que $F^{10}$ y por lo tanto $E(r)$ no dependen del tiempo y llegamos a 
\begin{equation}
    E_r = E_r(r)
\end{equation}

Si ahora tomamos de entrada el caso $\alpha=0$ 
\begin{equation}
\partial_1 F^{01} + \Gamma^0_{\mu\beta}F^{\mu\beta} + \Gamma^\beta_{\mu\beta}F^{0\mu} = 0
\end{equation}
Procediendo como antes podemos ver que el segundo término desaparece nuevamente pero el tercero no
\begin{equation}
\begin{split}
\Gamma^\beta_{\mu\beta} F^{0\mu} &= \Gamma^\beta_{1\beta} F^{01} = F^{01}(\Gamma^0_{10} +\Gamma^1_{11}+\Gamma^2_{12}+\Gamma^3_{13}) \\
&= F^{01}(\frac{A'}{2A} + \frac{B'}{2B} + \frac{2}{r}) = \frac{2}{r}F^{01} \Rightarrow \\
&\Rightarrow \frac{A'}{2A} + \frac{B'}{2B} = 0 \Rightarrow \frac{1}{2} \biggr(\frac{\partial}{\partial r}\ln(f)\biggr) = 0  
\end{split}
\end{equation}
porque sabíamos que $f(t)$ solo depende de $t$ y luego
\begin{equation}
    0 = \frac{\partial}{\partial r}F^{01} + \frac{2}{r}F^{01}
\end{equation}
Está última ecuación es fácil de resolver integrando 
\begin{equation}
    F^{01} = \frac{C}{r^2}
\end{equation}
donde $C$ es una constante de integración. Lo cual nos permite escribir
\begin{equation}
    E_r = \frac{C}{r^2}
\end{equation}
Podemos utilizar el teorema de Gauss en una superficie Gaussiana esférica y concluir que
\begin{equation}
    E_r = \frac{Q}{4\pi\epsilon_0 r^2}
    \label{Er}
\end{equation}
Esta es esencialmente la ley de Coulomb pero recordemos que $r$ es la circunferencia reducida que \textbf{no necesariamente} mide la distancia radial real en el espacio-tiempo que estamos analizando. 

Ahora solo nos falta expresar $A$ y $B$ como funciones de $r$ y terminamos. Partamos de la siguiente ecuación de Einstein
\begin{equation}
    R_{22} = \frac{8\pi G}{c^4} T_{22}
\end{equation}
Entonces
\begin{equation}
R_{22} = -\frac{r}{2B}\biggr(\frac{A'}{A} - \frac{B'}{B}\biggr) - \frac{1}{B}+1 = -\frac{1}{f}\frac{\partial}{\partial r}(rA) +1
\end{equation}
donde usamos que $f=AB$ y reglas de derivación. Ahora usamos la componente $T_{22}$ del tensor que ya calculamos (ec.\ref{componentestensorestresenergia}) y obtenemos
\begin{equation}
-\frac{1}{f} \frac{\partial}{\partial r}(rA) +1 = \frac{1}{f}\frac{8\pi G}{c^4} \frac{1}{2\mu_0 c^2} r^2 E_r^2
\end{equation}
Reemplazando $E_r$ por (ec. \ref{Er}) 
\begin{equation}
\frac{\partial}{\partial r} (rA) = f - \frac{GQ^2}{4\pi c^6 \mu_0 \epsilon_0^2r^2}
\end{equation}
Si ahora integramos y usamos que $c^2\mu_0= 1/\epsilon_0$ tenemos
\begin{equation}
    A = f + \frac{C_1(t)}{r} + \frac{GQ^2}{4\pi \epsilon_0 c^4 r^2}
    \label{A}
\end{equation}
Ahora, cuando $Q=0$, una de las suposiciones es que la métrica debe reducirse a la de Schwarzchild. Entonces, como mostramos en la ecuación (\ref{g00}), cuando $r$ es muy grande, entramos en límite de campo débil (Weak Field Limit) y $g_{00}$ tiende a 
\begin{equation*}
g_{00} = -1 - \frac{2GM}{c^2 r}
\end{equation*}
Entonces, en este límite, las geodésicas deben estar de acuerdo con el movimiento clásico gravitacional de Newton y mirando (ec. \ref{A}) se debe cumplir que $f=1$ y por lo tanto que 
\begin{equation}
C(t)= -\frac{2GM}{c^2} =- r_s
\end{equation}
que es el famoso \textbf{radio de Schwarzchild}. Además podemos definir 
\begin{equation}
    r_Q^2 = \frac{GQ^2}{4\pi\epsilon_0 c^4}
\end{equation}
y finalmente $A$ y $B$ quedan
\begin{equation}
\begin{split}
A &= 1 - \frac{r_s}{r} + \frac{r_Q^2}{r^2} \\
B &= \biggr(1 - \frac{r_s}{r} + \frac{r_Q^2}{r^2}\biggr)^{-1}
\end{split}
\end{equation}
y el tensor métrico es 

\begin{remarkbox}{Tensor métrico del espacio-tiempo de Reissner-Nordström}
\begin{equation*}
g_{\alpha\beta} = \begin{pmatrix}
\biggr(1 - \frac{r_s}{r} + \frac{r_Q^2}{r^2}\biggr) & 0 & 0 & 0 \\
0 & \biggr(1 - \frac{r_s}{r} + \frac{r_Q^2}{r^2}\biggr)^{-1} & 0 & 0\\
0 & 0 & -r^2 & 0 \\
0 & 0 & 0 & -r^2\sen^2{\theta} \\
\end{pmatrix}
\end{equation*}
\end{remarkbox}

\newthought{Hemos llegado} a la métrica de Reissner-Nordström completa derivada a partir de las ecuaciones de campo de Einstein junto con las ecuaciones de Maxwell. 


\subsection*{\textbf{Sobre las unidades geométricas}}
A partir de ahora, salvo que indiquemos lo contrario, usaremos las unidades geométricas. Esto es, tomaremos la velocidad de la luz $c$, y la constante de gravitación universal $G$ como :

\begin{remarkbox}{Consideración de las unidades geométricas}
\begin{equation*}
    c=G=1\ \ \textit{adimensional}
\end{equation*}
\end{remarkbox}

Hacemos esto para olvidarnos de las constantes, y hacer menos engorrosas las ecuaciones.
\begin{table}[h]
  \begin{center}
    \begin{tabular}{lccl}
      \toprule
      Variable & Unidades SI & Unidades Geom. & Factor \\
      \midrule
      Masa & $kg$ & $m$ &$c^2 G^{-1}$ \\
      Longitud & $m$ & $m$ & 1 \\
      Tiempo & $s$ & $m$ & $c^{-1}$\\
      Velocidad & $m s^{-1}$ & adim & $c$  \\
      Aceleración & $m s^{-2}$ & m{-1} & $c^2$  \\
      Fuerza & $kg m s^{-2}$ & adim & $c^4 G^{-1}$  \\
      Momento Angular & $kg m^2 s^{-1}$ & $m^2$ & $c^3 G^{-1}$ \\
      Momento & $kg m s^{-1}$ & $m$ & $c^3 G^{-1}$ \\
      Energía & $kg m^2 s^{-2}$ & $m$ & $c^{4} G^{-1}$\\
      Densidad de Energía & $kg m^{-1} s^{-2}$ & $m^{-2}$ & $c^4 G^{-1}$  \\
      \bottomrule
    \end{tabular}
  \end{center}
  \caption{Unidades Geométricas. Para convertir Geom. $\rightarrow$ SI, multiplicar por el factor. Para convertir SI $\rightarrow$ Geom., dividir por el factor. De forma general, para unidades SI de ''$kg^\alpha m^\beta s^\gamma$'', las unidades geométricas son ''$m^{\alpha + \beta + \gamma}$''.}
  \label{geounits}
\end{table}
\section{\huge{Geodésicas en la Geometría de Reissner-Nordström}}

\textcolor{myred}{\hrule}

\newthought{Ahora encontremos las ecuaciones que describen el movimiento de fotones y partículas no cargadas.} La partícula seguirá una geodésica \textbf{\textit{time-like}} mientras que el fotón una geodésica \textbf{\textit{light-like}}. Sea $x^\alpha = x^\alpha (\lambda)$ una curva parametrizada por $\lambda$, entonces debe cumplir la Ec. (\ref{geodesicx}):
\begin{equation}
    \frac{d^2 x^\alpha}{d \lambda^2} + \Gamma^\alpha_{\mu\nu} \frac{d x^\mu}{d \lambda} \frac{d x^\nu}{d \lambda} = 0
\end{equation}

Donde $\Gamma^\alpha_{\mu\nu}$ son los símbolos de Christoffel asociados a la métrica. Para una geodésica \textit{time-like} lo más natural es definir el parámetro $\lambda$ como el tiempo propio $\tau$, mientras que una geodésica \textit{light-like} no puede parametrizarse con $\tau$. Mantendremos por ahora el parámetro $\lambda$ para trabajar con los dos casos. Reemplazando los símbolos de Christoffel para nuestra métrica las ecuaciones de movimiento son:
 
\newthought{Para $\alpha=0$:}
\begin{equation}
    \frac{d^2 t}{d \lambda^2} + \frac{A^\prime}{A} \frac{d t}{d \lambda} \frac{d r}{d \lambda} = 0
\label{eqt3.1}
\end{equation}
\newthought{Para $\alpha=1$:}
\begin{equation}
\begin{split}
    \frac{d^2 r}{d \lambda^2} + \frac{A^\prime}{2B} \left(\frac{d t}{d \lambda}\right)^2 + \frac{B^\prime}{2B} \left(\frac{d r}{d \lambda}\right)^2 - \frac{r}{B} \left(\frac{d \theta}{d \lambda}\right)^2 &\\- \frac{r \sin^2{\theta}}{B} \left(\frac{d \phi}{d \lambda}\right)^2 &= 0
\label{eqradial3.1}
\end{split}
\end{equation}
\newthought{Para $\alpha=2$:}
\begin{equation}
    \frac{d^2 \theta}{d \lambda^2} + \frac{2}{r} \frac{d \theta}{d \lambda} \frac{d r}{d \lambda} - \sin{\theta}\cos{\theta} \left(\frac{d \phi}{d \lambda}\right)^2= 0
\end{equation}
\newthought{Para $\alpha=3$:}
\begin{equation}
    \frac{d^2 \phi}{d \lambda^2} + \frac{2}{r} \frac{d \phi}{d \lambda} \frac{d r}{d \lambda} - 2\cot{\theta}\frac{d \phi}{d \lambda} \frac{d \theta}{d \lambda}= 0
\label{eqazimutal3.1}
\end{equation}

Debido a la simetría esférica la trayectoria debe estar contenida en un plano definido por las condiciones iniciales\footnote{Trazamos el plano que contiene la velocidad en el instante inicial, ya sea de la partícula o del fotón. Sí la trayectoria se saliera de ese plano, la dirección en la que lo haga sería preferencial, rompiendo la simetría esférica del sistema.}, por lo que podemos poner sin ninguna pérdida de generalidad que $\theta=\pi/2$ en todo momento. Esto nos anula las derivadas de $\theta$, y las ecuaciones de movimiento se simplifican. Ahora, las Ecs. (\ref{eqradial3.1}) y (\ref{eqazimutal3.1}) nos quedan:

\begin{equation}
    \frac{d^2 r}{d \lambda^2} + \frac{A^\prime}{2B} \left(\frac{d t}{d \lambda}\right)^2 + \frac{B^\prime}{2B} \left(\frac{d r}{d \lambda}\right)^2 -  \frac{r}{B} \left(\frac{d \phi}{d \lambda}\right)^2 = 0
\label{eqradial3.2}
\end{equation}
\begin{equation}
    \frac{d^2 \phi}{d \lambda^2} + \frac{2}{r} \frac{d \phi}{d \lambda} \frac{d r}{d \lambda}= 0
\label{eqazimutal3.2}
\end{equation}

Sí dividimos la Ec. (\ref{eqazimutal3.2}) por $d\phi/d\lambda$, y usamos que:
\begin{equation}
\begin{split}
    \left(\frac{d\phi}{d\lambda}\right)^{-1} \frac{d^2\phi}{d\lambda^2} &= \frac{d}{d\lambda} \log{\left(\frac{d\phi}{d\lambda}\right)}\\
    \text{y, }\ \frac{2}{r} \frac{dr}{d\lambda} &= \frac{d}{d\lambda}\log{(r^2)}
\end{split}
\end{equation}
llegamos a que:

\begin{equation}
    \frac{d}{d\lambda}\log{\left(r^2\frac{d\phi}{d\lambda}\right)}=0 \Rightarrow r^2 \frac{d\phi}{d\lambda} = L = cte
\label{eqL}
\end{equation}

Donde $L$ es una constante del movimiento que coincide con el \textit{momento angular por unidad de masa} de la teoría Newtoniana. De manera similar, obtenemos para la Ec. (\ref{eqt3.1}) que:

\begin{equation}
    \frac{d}{d\lambda}\log{\left(A\frac{dt}{d\lambda}\right)}=0 \Rightarrow A \frac{dt}{d\lambda} = e = cte
\label{eqe}
\end{equation}

Con $e$ una constante del movimiento que se puede interpretar como la \textit{energía total relativista por unidad de masa}. Ahora usamos las Ecs. (\ref{eqL}) y (\ref{eqe}) en la Ec. (\ref{eqradial3.2}) y tenemos:

\begin{equation}
    \frac{d^2 r}{d \lambda^2} + \frac{A^\prime}{2B} \frac{e^2}{A^2} + \frac{B^\prime}{2B} \left(\frac{d r}{d \lambda}\right)^2 - \frac{r}{B} \frac{L^2}{r^4} = 0
\end{equation}
recordando que $B\prime=-A\prime/A^2$ y multiplicando por $2Bdr/d\lambda$:
\begin{equation}
\begin{split}
    0&=2B\frac{d^2 r}{d \lambda^2}\frac{dr}{d\lambda} - e^2 B^\prime \frac{dr}{d\lambda} + B^\prime \frac{dr}{d\lambda} \left(\frac{d r}{d \lambda}\right)^2 - \frac{2L^2}{r^3} \frac{dr}{d\lambda}\\
    &=\frac{d}{d\lambda}\left[ B \left(\frac{d r}{d \lambda}\right)^2 - e^2 B + \frac{L^2}{r^2}\right]
\end{split}
\end{equation}
por lo que:
\begin{equation}
    B \left(\frac{d r}{d \lambda}\right)^2 - e^2 B + \frac{L^2}{r^2} = -(e_0)^2 = cte
\label{eqe0}
\end{equation}

Donde $e_0$ se puede pensar como la \textit{energía total en reposo por unidad de masa}. Reescribiendo la Ec. (\ref{eqe0}) obtenemos:
\begin{equation}
    \left(\frac{d r}{d \lambda}\right)^2 = e^2 -A\left( \frac{L^2}{r^2} + {e_0}^2\right)
\label{eqradial3.3}
\end{equation}

Esta ecuación nos da $dr/d\lambda$ en función de $r$. Para obtener una ecuación de $dr/d\phi$ dividimos la Ec. (\ref{eqradial3.3}) por $(d\phi/d\lambda)^2 = L^2/r^4$:
\begin{equation}
    \left( \frac{dr}{d\phi} \right)^2 = \frac{r^4 e^2}{L^2} - r^2 A \left(1 + \frac{r^2 {e_0}^2}{L^2} \right)
\end{equation}
recordando que $A=1 - (r_s/r) + (r_Q/r)^2$, llegamos a:
\begin{equation}
\begin{split}
    \left( \frac{dr}{d\phi} \right)^2 = -{r_Q}^2 + r_s r - \left(1 + \frac{{r_Q}^2 {e_0}^2}{L^2} \right) r^2 &\\+ \frac{r_s {e_0}^2}{L_2} r^3 &- \frac{{e_0}^2 - e^2}{L^2} r^4
\end{split}
\label{eqrposta}
\end{equation}

Esta es la ecuación que buscamos, nos da $dr/d\phi$ en términos de $r$. En teoría ya con esto podemos encontrar la trayectoria de una partícula o fotón, tomando los valores de las constantes a partir de las condiciones iniciales. Sin embargo, la Ec. (\ref{eqrposta}) se puede simplificar aún más apelando a la métrica\footnote{Recordemos que $c=G=1$ y $\theta=\pi/2$.}:

\begin{equation}
    ds^2 = d\tau^2 = A dt^2 - \frac{1}{A} dr^2 - r^2 d\phi^2
\end{equation}

De las Ecs. (\ref{eqL}), (\ref{eqe}) y (\ref{eqe0}) tenemos:
\begin{equation}
\begin{split}
    d\phi^2 &= \frac{L^2}{r^4} d\lambda^2\\
    dt^2 &= \frac{e^2}{A^2} d\lambda^2\\
    dr^2 &= \left[e^2 - A\left({e_0}^2 + \frac{L^2}{r^2}\right)\right] d\lambda^2
\end{split}
\end{equation}

Sustituyendo $d\phi^2$, $dt^2$ y $dr^2$ en la métrica conseguimos:

\begin{remarkbox}{Relación entre $d\tau^2$ y $d\lambda^2$.}
\begin{equation}
    d\tau^2 = {e_0}^2 d\lambda^2
\label{doujou}
\end{equation}
\end{remarkbox}

Finalmente, sí tenemos una partícula con masa y parametrizamos con el tiempo propio, $d\lambda^2=d\tau^2$, según la Ec. (\ref{doujou}) tendremos que $e_0=1$. Sí tenemos un fotón el tiempo propio es nulo, $d\tau^2=0$, y por lo tanto $e_0=0$. Usando este resultado, junto a la Ec. (\ref{eqrposta}) escribimos al fin la ecuación de la trayectoria de un cuerpo no cargado:

\newthought{Para la luz:}
\begin{equation}
    \left( \frac{dr}{d\phi} \right)^2 = -{r_Q}^2 + r_s r - r^2 - \frac{e^2}{L^2} r^4
\label{eqlightlike}
\end{equation}
\newthought{Para un cuerpo, no cargado, con masa:}
\begin{equation}
    \left( \frac{dr}{d\phi} \right)^2 = -{r_Q}^2 + r_s r - \left(1 + \frac{{r_Q}^2}{L^2} \right) r^2 + \frac{r_s}{L_2} r^3 - \frac{1 - e^2}{L^2} r^4
\label{eqtimelike}
\end{equation}

\subsection*{\textbf{Trayectoria de un cuerpo cargado en la Geometría de R-N.}}
\newthought{Teniendo el espacio-tiempo \textit{lleno} de campo electromagnético, es natural preguntarse} la trayectoria de una carga. Para ello nos dirigimos al formalismo de Lagrange. Para una partícula cargada, en un espacio-tiempo cuya curvatura está descrita por $g_{\mu\nu}$, el lagrangiano es\footnote{Sí no hubiera campo electromagnético, $A_\mu=0$ y la solución se correspondería con el lagrangiano de una partícula bajo influencia de un campo gravitatorio. El término correspondiente a la ''energía potencial'' está incluido en el término con la métrica. En espacio plano, sin campos, se reduce a la energía cinética de una partícula en el vacío.}:

\begin{equation}
    \mathcal{L} = \frac{1}{2} g_{\mu\nu} \dot{x^\mu}\dot{x^\nu} + q A_\mu \dot{x^\mu}
\end{equation}

Donde $q$ es la \textit{carga por unidad de masa} de la partícula, $A_\mu$ es el cuadri-potencial electromagnético y los ''puntos'' representan una derivada respecto al tiempo propio $\tau$ de la partícula. En nuestro caso solo tenemos campo eléctrico radial, y el cuadri-potencial resulta ser:
  
\begin{equation}
    A_\mu=[A_0,A_1,A_2,A_3]=\left[\frac{Q}{4\pi\epsilon_0 r^2},\ 0,\ 0,\ 0 \right]
\end{equation}

Y, usando la métrica de Reissner-Nordström, el lagrangiano es:

\begin{equation}
    \mathcal{L} = \frac{1}{2}\left( A\dot{t}^2 - \frac{1}{A} \dot{r}^2 - r^2 \dot{\theta}^2 - r^2 \sin^2{\theta} \dot{\phi}^2 \right) + q \frac{Q}{4\pi\epsilon_0 r^2} \dot{t}
\end{equation}

Como antes la trayectoria será en un plano, así que podemos sin pérdida de generalidad considerar $\theta=\pi/2$ y $\dot{\theta}=0$. Luego, las ecuaciones de Euler-Lagrange son:

\newthought{Para $t$:}
\begin{equation}
\frac{d}{d\tau}\frac{\partial \mathcal{L}}{\partial \dot{t}} - \frac{\partial \mathcal{L}}{\partial t} = \frac{d}{d\tau}\left[ A\dot{t} + \frac{qQ}{4\pi\epsilon_0 r^2} \right]  = 0
\label{eqt3.4}
\end{equation}
\newthought{Para $r$:}
\begin{equation}
    \frac{d}{d\tau}\frac{\partial \mathcal{L}}{\partial \dot{r}} - \frac{\partial \mathcal{L}}{\partial r} = -\frac{\ddot{r}}{A} - \frac{A^\prime}{2} \dot{t}^2 - 2 r \dot{\phi}^2 + \frac{qQ}{2\pi\epsilon_0 r^3}  = 0
\label{eqradial3.4}
\end{equation}
\newthought{Para $\phi$:}
\begin{equation}
    \frac{d}{d\tau}\frac{\partial \mathcal{L}}{\partial \dot{\phi}} - \frac{\partial \mathcal{L}}{\partial \phi} =\frac{d}{d \tau}\left[ r^2 \dot{\phi} \right] = 0
\label{eqazimutal3.4}
\end{equation}

De las Ecs. (\ref{eqt3.4}) y (\ref{eqazimutal3.4}) tenemos las constantes de movimiento:

\begin{equation}
    A\dot{t} + \frac{qQ}{4\pi\epsilon_0 r} = e\ \ \text{, y }\ r^2 \dot{\phi} = L
\label{conservposta}
\end{equation}
la única diferencia de las constantes con el caso de la partícula sin carga, es el término $e$ que aparece la energía potencial electrostática $\frac{qQ}{4\pi\epsilon_0 r}$. Ahora, para llegar a una expresión de $dr/d\phi$ como función de $r$, necesitamos apelar a la métrica:

\begin{equation}
    ds^2=d\tau^2=Adt^2 - \frac{1}{A}dr^2- r^2d\phi^2
\end{equation}
Dividimos por $d\tau^2$, y multiplicamos por $A$:
\begin{equation}
    \dot{r}^2 + A - A^2 \dot{t}^2 + A r^2 \dot{\phi}^2 =0
\end{equation}

Usando las Ecs. (\ref{conservposta}), sustituimos $\dot{\phi}$ y $\dot{t}$:
\begin{equation}
    \dot{r}^2 + A - \left(e-\frac{qQ}{4\pi\epsilon_0 r}\right)^2 + A \frac{L^2}{r^2} =0
\end{equation}

Finalmente, dividiendo por $\dot{\phi}$:

\newthought{Para un cuerpo cargado con masa:}
\begin{equation}
\begin{split}
    \left( \frac{dr}{d\phi} \right)^2 &= -r^2 A\left(1+\frac{r^2}{L^2}\right) + \frac{r^4}{L^2}\left( e - \frac{qQ}{4\pi\epsilon_0 r} \right)^2\\
    &=-{r_Q}^2 + r_s r - \left( 1 + \frac{{r_Q}^2 - \left(\frac{qQ}{4\pi\epsilon_0}\right)^2}{L^2} \right) r^2 \\ &+ \frac{1}{L^2} \left(r_s - 2e\frac{qQ}{4\pi\epsilon_0}\right)r^3 - \frac{(1-e^2)}{L^2}r^4
\end{split}
\label{eqtimelikecargado}
\end{equation}

Esta es la ecuación que describe la órbita de una partícula cargada en el espacio-tiempo de Reissner-Nordström. Como era esperado, se reduce a la Ec. (\ref{eqtimelike}) cuándo hacemos $q=0$.%The Reissner-Nordstrom metric
%\chapter{\textcolor{myred}{La métrica de Reissner-Nordström}}

%Principio de equivalencia (fuerte): En un laboratorio en caída libre (no rotante) que ocupa una pequeña región del espacio tiempo, las leyes de la física son las correspondientes a la relatividad especial
\subsection*{\textbf{Sobre las unidades geométricas}}
A partir de ahora, salvo que indiquemos lo contrario, usaremos las unidades geométricas. Esto es, tomaremos la velocidad de la luz $c$, y la constante de gravitación universal $G$ como :

\begin{remarkbox}{Consideración de las unidades geométricas}
\begin{equation*}
    c=G=1\ \ \textit{adimensional}
\end{equation*}
\end{remarkbox}

Hacemos esto para olvidarnos de las constantes, y hacer menos engorrosas las ecuaciones.
\begin{table}[h]
  \begin{center}
    \begin{tabular}{lccl}
      \toprule
      Variable & Unidades SI & Unidades Geom. & Factor \\
      \midrule
      Masa & $kg$ & $m$ &$c^2 G^{-1}$ \\
      Longitud & $m$ & $m$ & 1 \\
      Tiempo & $s$ & $m$ & $c^{-1}$\\
      Velocidad & $m s^{-1}$ & adim & $c$  \\
      Aceleración & $m s^{-2}$ & m{-1} & $c^2$  \\
      Fuerza & $kg m s^{-2}$ & adim & $c^4 G^{-1}$  \\
      Momento Angular & $kg m^2 s^{-1}$ & $m^2$ & $c^3 G^{-1}$ \\
      Momento & $kg m s^{-1}$ & $m$ & $c^3 G^{-1}$ \\
      Energía & $kg m^2 s^{-2}$ & $m$ & $c^{4} G^{-1}$\\
      Densidad de Energía & $kg m^{-1} s^{-2}$ & $m^{-2}$ & $c^4 G^{-1}$  \\
      \bottomrule
    \end{tabular}
  \end{center}
  \caption{Unidades Geométricas. Para convertir Geom. $\rightarrow$ SI, multiplicar por el factor. Para convertir SI $\rightarrow$ Geom., dividir por el factor. De forma general, para unidades SI de "$kg^\alpha m^\beta s^\gamma$", las unidades geométricas son "$m^{\alpha + \beta + \gamma}$".}
  \label{geounits}
\end{table}
\section{\huge{Geodésicas en la Geometría de Reissner-Nordström}}

\textcolor{myred}{\hrule}

\newthought{Ahora encontremos las ecuaciones que describen el movimiento de fotones y partículas no cargadas.} La partícula seguirá una geodésica \textbf{\textit{time-like}} mientras que el fotón una geodésica \textbf{\textit{light-like}}. Sea $x^\alpha = x^\alpha (\lambda)$ una curva parametrizada por $\lambda$, entonces debe cumplir la Ec. (\ref{geodesicx}):
\begin{equation}
    \frac{d^2 x^\alpha}{d \lambda^2} + \Gamma^\alpha_{\mu\nu} \frac{d x^\mu}{d \lambda} \frac{d x^\nu}{d \lambda} = 0
\end{equation}

Donde $\Gamma^\alpha_{\mu\nu}$ son los símbolos de Christoffel asociados a la métrica. Para una geodésica \textit{time-like} lo más natural es definir el parámetro $\lambda$ como el tiempo propio $\tau$, mientras que una geodésica \textit{light-like} no puede parametrizarse con $\tau$. Mantendremos por ahora el parámetro $\lambda$ para trabajar con los dos casos. Reemplazando los símbolos de Christoffel para nuestra métrica las ecuaciones de movimiento son:
 
\newthought{Para $\alpha=0$:}
\begin{equation}
    \frac{d^2 t}{d \lambda^2} + \frac{A^\prime}{A} \frac{d t}{d \lambda} \frac{d r}{d \lambda} = 0
\label{eqt3.1}
\end{equation}
\newthought{Para $\alpha=1$:}
\begin{equation}
\begin{split}
    \frac{d^2 r}{d \lambda^2} + \frac{A^\prime}{2B} \left(\frac{d t}{d \lambda}\right)^2 + \frac{B^\prime}{2B} \left(\frac{d r}{d \lambda}\right)^2 - \frac{r}{B} \left(\frac{d \theta}{d \lambda}\right)^2 &\\- \frac{r \sin^2{\theta}}{B} \left(\frac{d \phi}{d \lambda}\right)^2 &= 0
\label{eqradial3.1}
\end{split}
\end{equation}
\newthought{Para $\alpha=2$:}
\begin{equation}
    \frac{d^2 \theta}{d \lambda^2} + \frac{2}{r} \frac{d \theta}{d \lambda} \frac{d r}{d \lambda} - \sin{\theta}\cos{\theta} \left(\frac{d \phi}{d \lambda}\right)^2= 0
\end{equation}
\newthought{Para $\alpha=3$:}
\begin{equation}
    \frac{d^2 \phi}{d \lambda^2} + \frac{2}{r} \frac{d \phi}{d \lambda} \frac{d r}{d \lambda} - 2\cot{\theta}\frac{d \phi}{d \lambda} \frac{d \theta}{d \lambda}= 0
\label{eqazimutal3.1}
\end{equation}

Debido a la simetría esférica la trayectoria debe estar contenida en un plano definido por las condiciones iniciales\footnote{Trazamos el plano que contiene la velocidad en el instante inicial, ya sea de la partícula o del fotón. Sí la trayectoria se saliera de ese plano, la dirección en la que lo haga sería preferencial, rompiendo la simetría esférica del sistema.}, por lo que podemos poner sin ninguna pérdida de generalidad que $\theta=\pi/2$ en todo momento. Esto nos anula las derivadas de $\theta$, y las ecuaciones de movimiento se simplifican. Ahora, las Ecs. (\ref{eqradial3.1}) y (\ref{eqazimutal3.1}) nos quedan:

\begin{equation}
    \frac{d^2 r}{d \lambda^2} + \frac{A^\prime}{2B} \left(\frac{d t}{d \lambda}\right)^2 + \frac{B^\prime}{2B} \left(\frac{d r}{d \lambda}\right)^2 -  \frac{r}{B} \left(\frac{d \phi}{d \lambda}\right)^2 = 0
\label{eqradial3.2}
\end{equation}
\begin{equation}
    \frac{d^2 \phi}{d \lambda^2} + \frac{2}{r} \frac{d \phi}{d \lambda} \frac{d r}{d \lambda}= 0
\label{eqazimutal3.2}
\end{equation}

Sí dividimos la Ec. (\ref{eqazimutal3.2}) por $d\phi/d\lambda$, y usamos que:
\begin{equation}
\begin{split}
    \left(\frac{d\phi}{d\lambda}\right)^{-1} \frac{d^2\phi}{d\lambda^2} &= \frac{d}{d\lambda} \log{\left(\frac{d\phi}{d\lambda}\right)}\\
    \text{y, }\ \frac{2}{r} \frac{dr}{d\lambda} &= \frac{d}{d\lambda}\log{(r^2)}
\end{split}
\end{equation}
llegamos a que:

\begin{equation}
    \frac{d}{d\lambda}\log{\left(r^2\frac{d\phi}{d\lambda}\right)}=0 \Rightarrow r^2 \frac{d\phi}{d\lambda} = L = cte
\label{eqL}
\end{equation}

Donde $L$ es una constante del movimiento que coincide con el \textit{momento angular por unidad de masa} de la teoría Newtoniana. De manera similar, obtenemos para la Ec. (\ref{eqt3.1}) que:

\begin{equation}
    \frac{d}{d\lambda}\log{\left(A\frac{dt}{d\lambda}\right)}=0 \Rightarrow A \frac{dt}{d\lambda} = e = cte
\label{eqe}
\end{equation}

Con $e$ una constante del movimiento que se puede interpretar como la \textit{energía total relativista por unidad de masa}. Ahora usamos las Ecs. (\ref{eqL}) y (\ref{eqe}) en la Ec. (\ref{eqradial3.2}) y tenemos:

\begin{equation}
    \frac{d^2 r}{d \lambda^2} + \frac{A^\prime}{2B} \frac{e^2}{A^2} + \frac{B^\prime}{2B} \left(\frac{d r}{d \lambda}\right)^2 - \frac{r}{B} \frac{L^2}{r^4} = 0
\end{equation}
recordando que $B\prime=-A\prime/A^2$ y multiplicando por $2Bdr/d\lambda$:
\begin{equation}
\begin{split}
    0&=2B\frac{d^2 r}{d \lambda^2}\frac{dr}{d\lambda} - e^2 B^\prime \frac{dr}{d\lambda} + B^\prime \frac{dr}{d\lambda} \left(\frac{d r}{d \lambda}\right)^2 - \frac{2L^2}{r^3} \frac{dr}{d\lambda}\\
    &=\frac{d}{d\lambda}\left[ B \left(\frac{d r}{d \lambda}\right)^2 - e^2 B + \frac{L^2}{r^2}\right]
\end{split}
\end{equation}
por lo que:
\begin{equation}
    B \left(\frac{d r}{d \lambda}\right)^2 - e^2 B + \frac{L^2}{r^2} = -(e_0)^2 = cte
\label{eqe0}
\end{equation}

Donde $e_0$ se puede pensar como la \textit{energía total en reposo por unidad de masa}. Reescribiendo la Ec. (\ref{eqe0}) obtenemos:
\begin{equation}
    \left(\frac{d r}{d \lambda}\right)^2 = e^2 -A\left( \frac{L^2}{r^2} + {e_0}^2\right)
\label{eqradial3.3}
\end{equation}

Esta ecuación nos da $dr/d\lambda$ en función de $r$. Para obtener una ecuación de $dr/d\phi$ dividimos la Ec. (\ref{eqradial3.3}) por $(d\phi/d\lambda)^2 = L^2/r^4$:
\begin{equation}
    \left( \frac{dr}{d\phi} \right)^2 = \frac{r^4 e^2}{L^2} - r^2 A \left(1 + \frac{r^2 {e_0}^2}{L^2} \right)
\end{equation}
recordando que $A=1 - (r_s/r) + (r_Q/r)^2$, llegamos a:
\begin{equation}
\begin{split}
    \left( \frac{dr}{d\phi} \right)^2 = -{r_Q}^2 + r_s r - \left(1 + \frac{{r_Q}^2 {e_0}^2}{L^2} \right) r^2 &\\+ \frac{r_s {e_0}^2}{L_2} r^3 &- \frac{{e_0}^2 - e^2}{L^2} r^4
\end{split}
\label{eqrposta}
\end{equation}

Esta es la ecuación que buscamos, nos da $dr/d\phi$ en términos de $r$. En teoría ya con esto podemos encontrar la trayectoria de una partícula o fotón, tomando los valores de las constantes a partir de las condiciones iniciales. Sin embargo, la Ec. (\ref{eqrposta}) se puede simplificar aún más apelando a la métrica\footnote{Recordemos que $c=G=1$ y $\theta=\pi/2$.}:

\begin{equation}
    ds^2 = d\tau^2 = A dt^2 - \frac{1}{A} dr^2 - r^2 d\phi^2
\end{equation}

De las Ecs. (\ref{eqL}), (\ref{eqe}) y (\ref{eq0}) tenemos:
\begin{equation}
\begin{split}
    d\phi^2 &= \frac{L^2}{r^4} d\lambda^2\\
    dt^2 &= \frac{e^2}{A^2} d\lambda^2\\
    dr^2 &= \left[e^2 - A\left({e_0}^2 + \frac{L^2}{r^2}\right)\right] d\lambda^2
\end{split}
\end{equation}

Sustituyendo $d\phi^2$, $dt^2$ y $dr^2$ en la métrica conseguimos:

\begin{remarkbox}{Relación entre $d\tau^2$ y $d\lambda^2$.}
\begin{equation}
    d\tau^2 = {e_0}^2 d\lambda^2
\label{doujou}
\end{equation}
\end{remarkbox}

Finalmente, sí tenemos una partícula con masa y parametrizamos con el tiempo propio, $d\lambda^2=d\tau^2$, según la Ec. (\ref{doujou}) tendremos que $e_0=1$. Sí tenemos un fotón el tiempo propio es nulo, $d\tau^2=0$, y por lo tanto $e_0=0$. Usando este resultado, junto a la Ec. (\ref{eqrposta}) escribimos al fin la ecuación de la trayectoria de un cuerpo no cargado:

\newthought{Para la luz:}
\begin{equation}
    \left( \frac{dr}{d\phi} \right)^2 = -{r_Q}^2 + r_s r - r^2 - \frac{e^2}{L^2} r^4
\label{eqlightlike}
\end{equation}
\newthought{Para un cuerpo, no cargado, con masa:}
\begin{equation}
    \left( \frac{dr}{d\phi} \right)^2 = -{r_Q}^2 + r_s r - \left(1 + \frac{{r_Q}^2}{L^2} \right) r^2 + \frac{r_s}{L_2} r^3 - \frac{1 - e^2}{L^2} r^4
\label{eqtimelike}
\end{equation}

\subsection*{\textbf{Trayectoria de un cuerpo cargado en la Geometría de R-N.}}
\newthought{Teniendo el espacio-tiempo \textit{lleno} de campo electromagnético, es natural preguntarse} la trayectoria de una carga. Para ello nos dirigimos al formalismo de Lagrange. Para una partícula cargada, en un espacio-tiempo cuya curvatura está descrita por $g_{\mu\nu}$, el lagrangiano es\footnote{Sí no hubiera campo electromagnético, $A_\mu=0$ y la solución se correspondería con el lagrangiano de una partícula bajo influencia de un campo gravitatorio. El término correspondiente a la "energía potencial" está incluido en el término con la métrica. En espacio plano, sin campos, se reduce a la energía cinética de una partícula en el vacío.}:

\begin{equation}
    \mathcal{L} = \frac{1}{2} g_{\mu\nu} \dot{x^\mu}\dot{x^\nu} + q A_\mu \dot{x^\mu}
\end{equation}

Donde $q$ es la \textit{carga por unidad de masa} de la partícula, $A_\mu$ es el cuadri-potencial electromagnético y los "puntos" representan una derivada respecto al tiempo propio $\tau$ de la partícula. En nuestro caso solo tenemos campo eléctrico radial, y el cuadri-potencial resulta ser:
  
\begin{equation}
    A_\mu=[A_0,A_1,A_2,A_3]=\left[\frac{Q}{4\pi\epsilon_0 r^2},\ 0,\ 0,\ 0 \right]
\end{equation}

Y, usando la métrica de Reissner-Nordström, el lagrangiano es:

\begin{equation}
    \mathcal{L} = \frac{1}{2}\left( A\dot{t}^2 - \frac{1}{A} \dot{r}^2 - r^2 \dot{\theta}^2 - r^2 \sin^2{\theta} \dot{\phi}^2 \right) + q \frac{Q}{4\pi\epsilon_0 r^2} \dot{t}
\end{equation}

Como antes la trayectoria será en un plano, así que podemos sin pérdida de generalidad considerar $\theta=\pi/2$ y $\dot{\theta}=0$. Luego, las ecuaciones de Euler-Lagrange son:

\newthought{Para $t$:}
\begin{equation}
\frac{d}{d\tau}\frac{\partial \mathcal{L}}{\partial \dot{t}} - \frac{\partial \mathcal{L}}{\partial t} = \frac{d}{d\tau}\left[ A\dot{t} + \frac{qQ}{4\pi\epsilon_0 r^2} \right]  = 0
\label{eqt3.4}
\end{equation}
\newthought{Para $r$:}
\begin{equation}
    \frac{d}{d\tau}\frac{\partial \mathcal{L}}{\partial \dot{r}} - \frac{\partial \mathcal{L}}{\partial r} = -\frac{\ddot{r}}{A} - \frac{A^\prime}{2} \dot{t}^2 - 2 r \dot{\phi}^2 + \frac{qQ}{2\pi\epsilon_0 r^3}  = 0
\label{eqradial3.4}
\end{equation}
\newthought{Para $\phi$:}
\begin{equation}
    \frac{d}{d\tau}\frac{\partial \mathcal{L}}{\partial \dot{\phi}} - \frac{\partial \mathcal{L}}{\partial \phi} =\frac{d}{d \tau}\left[ r^2 \dot{\phi} \right] = 0
\label{eqazimutal3.4}
\end{equation}

De las Ecs. (\ref{eqt3.4}) y (\ref{eqazimutal3.4}) tenemos las constantes de movimiento:

\begin{equation}
    A\dot{t} + \frac{qQ}{4\pi\epsilon_0 r} = e\ \ \text{, y }\ r^2 \dot{\phi} = L
\label{conservposta}
\end{equation}
la única diferencia de las constantes con el caso de la partícula sin carga, es el término $e$ que aparece la energía potencial electrostática $\frac{qQ}{4\pi\epsilon_0 r}$. Ahora, para llegar a una expresión de $dr/d\phi$ como función de $r$, necesitamos apelar a la métrica:

\begin{equation}
    ds^2=d\tau^2=Adt^2 - \frac{1}{A}dr^2- r^2d\phi^2
\end{equation}
Dividimos por $d\tau^2$, y multiplicamos por $A$:
\begin{equation}
    \dot{r}^2 + A - A^2 \dot{t}^2 + A r^2 \dot{\phi}^2 =0
\end{equation}

Usando las Ecs. (\ref{conservposta}), sustituimos $\dot{\phi}$ y $\dot{t}$:
\begin{equation}
    \dot{r}^2 + A - \left(e-\frac{qQ}{4\pi\epsilon_0 r}\right)^2 + A \frac{L^2}{r^2} =0
\end{equation}

Finalmente, dividiendo por $\dot{\phi}$:

\newthought{Para un cuerpo cargado con masa:}
\begin{equation}
\begin{split}
    \left( \frac{dr}{d\phi} \right)^2 &= -r^2 A\left(1+\frac{r^2}{L^2}\right) + \frac{r^4}{L^2}\left( e - \frac{qQ}{4\pi\epsilon_0 r} \right)^2\\
    &=-{r_Q}^2 + r_s r - \left( 1 + \frac{{r_Q}^2 - \left(\frac{qQ}{4\pi\epsilon_0 r}\right)^2}{L^2} \right) r^2 \\ &+ \frac{1}{L^2} \left(r_s - 2e\frac{qQ}{4\pi\epsilon_0 r}\right)r^3 - \frac{(1-e^2)}{L^2}r^4
\end{split}
\end{equation}

Esta es la ecuación que describe la órbita de una partícula cargada en el espacio-tiempo de Reissner-Nordström. Como era esperado, se reduce a la Ec. (\ref{eqtimelike}) cuándo hacemos $q=0$.

%\subsection*{\textbf{Horizonte de Eventos}}

%\newthought{Para encontrar singularidades} tenemos que mirar donde las componentes de la métrica no están definidas. Para la métrica de R-N, la componente crucial es:

%\begin{equation}
%    g_{tt}=-\frac{1}{g_{rr}} = A = 1 - \frac{r_s}{r} + \frac{{r_Q}^2}{r^2}
%\end{equation}

%Los puntos singulares de la métrica son $r=0$ y $r$ tal que $A=1/g_{rr}=0$. Calculemos las raíces de $A$:
%\begin{equation}
%    A=1 - \frac{r_s}{r} + \frac{{r_Q}^2}{r^2}=0 \Rightarrow r^2 - r_s r + (r_Q)^2 = 0
%\end{equation}
%cuyas soluciones son:
%\begin{equation}
%    r_\pm = \frac{1}{2}\left(r_s \pm \sqrt{{r_s}^2 - (2r_Q)^2}\right)
%\label{eqhorizonte}
%\end{equation}

%Entonces tenemos tres puntos singulares: $r=r_+$, $r=r_-$ y $r=0$. Dependiendo de los valores de $r_s$ y $r_Q$ puede que la Ec. (\ref{eqhorizonte}) tenga dos, una o ninguna solución. Esto nos motiva a analizar tres casos distintos:

%\newthought{Para $r_s>2r_Q$} tenemos dos raíces distintas $r_+$ y $r_-$. Llamamos $r_+$ el horizonte de eventos porque para radios menores a ese, ni la luz puede escaparse del mismo. El nombre de "agujero negro" se refiere entonces a un cuerpo cuyas dimensiones sean menores a $r_+$\footnote{Recordemos que todo lo que estudiamos de la métrica de R-N vale solo para una región del espacio vacío. Todo lo que esté "dentro" del cuerpo yace por fuera del alcance de la métrica que desarrollamos.}. Podemos dividir el espacio en tres regiones:
%\begin{itemize}
%    \item Región 1: $r_+<r<\infty$
%    \item Región 2: $r_-<r<\infty$
%    \item Región 3: $0<r<r_-$
%\end{itemize}

%Se puede demostrar que un cuerpo en la región 2 está obligado a disminuir su coordenada $r$, al menos hasta entrar en la región 3 donde eso deja de ser cierto y el cuerpo no está obligado a estrellarse con la singularidad en $r=0$.
%\chapter{\textcolor{myred}{Simulaciones Computacionales}}

\section{\huge{Resolviendo las Geodésicas mediante Métodos Númericos}}

\textcolor{myred}{\hrule}
\begin{flushright}
\textit{What happens if a big asteroid hits Earth?\\Judging from realistic simulations involving a sledge hammer\\and a common laboratory frog, we can assume it will be pretty bad.\\\textbf{Dave Barry}}
\end{flushright}
 
\newthought{En los capítulos anteriores} vimos los conceptos, encontramos la métrica y derivamos las geodésicas. El trabajo está hecho, a partir de aquí solo jugaremos con las ideas que ya hablamos.



\newthought{El objetivo es simular la trayectoria} de objetos masivos con carga $q$, y de fotones, en los alrededores de un objeto esférico de masa $M$ y carga $Q$ utilizando el método de Runge-Kutta de 4to orden. Para eso recurrimos a las ecuaciones geodésicas (\ref{eqlightlike}), y (\ref{eqtimelikecargado}) y notamos dos inconvenientes:

\begin{itemize}
    \item Son ecuaciones diferenciales de primer orden, pero no son lineales.
    \item Nos hablan de $dr/d\phi$, lo cuál nos permitiría dibujar la órbita pero se pierde la noción de ''velocidad''\footnote{Para el caso de cuerpos con masa, nos referimos a las variaciones de las coordenadas, $t$, $r$ y $\phi$ respecto de el tiempo propio $\tau$. Para el caso de la luz, es más delicado porque sería importante definir una coordenada temporal específica.} del objeto.
\end{itemize}

Lo ideal sería contar con un conjunto de ecuaciones lineales de segundo orden de $r=r(\phi)$ para la órbita, y de las coordenadas $t=t(\tau)$, $r=r(\tau)$ y $\phi=\phi(\tau)$. Para estas últimas, podemos usar las Ecs. (\ref{eqt3.4}), (\ref{eqradial3.4}) y (\ref{eqazimutal3.4}) obtenidas del Lagrangiano de una partícula cargada en el espacio-tiempo de R-N. Idealmente, nuestra simulación nos permitiría animar la trayectoria de cuerpos con masa y carga, y dibujar la trayectoria cualquier cuerpo, incluso fotones. Animarla daría una noción de ''velocidad'', y dibujarla nos podría representar fenómenos físicos como la precesión de perihelios y afelios, las órbitas elípticas de planetas, como un agujero negro ''atrapa'' la luz, dilatación temporal en las cercanías de un campo gravitatorio intenso, etc.

\subsection*{\textbf{Ecuación general de la trayectoria de \textit{cosas} en la geometría R-N}}
Para obtener ecuaciones de segundo orden, hagamos lo más directo: derivemos las ecuaciones que tenemos. Empecemos con (\ref{eqtimelikecargado}):

\begin{equation}
\begin{split}
    \left( \frac{dr}{d\phi} \right)^2 &=-{r_Q}^2 + r_s r - \left( 1 + \frac{{r_Q}^2 - \left(\frac{qQ}{4\pi\epsilon_0 r}\right)^2}{L^2} \right) r^2 \\ &+ \frac{1}{L^2} \left(r_s - 2e\frac{qQ}{4\pi\epsilon_0}\right)r^3 - \frac{(1-e^2)}{L^2}r^4
\end{split}
\end{equation}
derivando respecto a $\phi$ a ambos miembros:

\begin{equation}
\begin{split}
    2 \frac{dr}{d\phi} \frac{d^2r}{d\phi^2} &=r_s \frac{dr}{d\phi} - 2 \left( 1 + \frac{{r_Q}^2 - \left(\frac{qQ}{4\pi\epsilon_0}\right)^2}{L^2} \right) r \frac{dr}{d\phi} \\ &+ \frac{3}{L^2} \left(r_s - 2e\frac{qQ}{4\pi\epsilon_0}\right)r^2\frac{dr}{d\phi} - \frac{4(1-e^2)}{L^2}r^3\frac{dr}{d\phi}
\end{split}
\end{equation}
y dividiendo por $2dr/d\phi$ a todo, tenemos:
\begin{equation}
\begin{split}
    \frac{d^2r}{d\phi^2} &= \frac{r_s}{2} - \left( 1 + \frac{{r_Q}^2 - \left(\frac{qQ}{4\pi\epsilon_0}\right)^2}{L^2} \right) r \\ &+ \frac{3}{2 L^2} \left(r_s - 2e\frac{qQ}{4\pi\epsilon_0}\right)r^2 - \frac{2(1-e^2)}{L^2}r^3
\end{split}
\label{eqrsegunda}
\end{equation}

Introducimos ahora el cambio de variables $u=1/r$. Aplicando regla de la cadena:
\begin{equation}
    \frac{d^2 u}{d \phi^2} = \frac{d}{d \phi} \left[ \frac{du}{d\phi} \right] = \frac{d}{d \phi} \left[ -\frac{1}{r^2} \frac{dr}{d\phi} \right] = \frac{2}{r^3} \left( \frac{dr}{d\phi} \right)^2 - \frac{1}{r^2} \frac{d^2 r}{d\phi^2}
\end{equation}

Reemplazando el primer término según la Ec. (\ref{eqtimelikecargado}) y el segundo término con la Ec. (\ref{eqrsegunda}) conseguimos:
\begin{equation}
\begin{split}
    \frac{d^2 u}{d \phi^2} &= \frac{2}{r^3} \Big[ -{r_Q}^2 + r_s r - \left( 1 + \frac{{r_Q}^2 - \left(\frac{qQ}{4\pi\epsilon_0 r}\right)^2}{L^2} \right) r^2 \\&+ \frac{1}{L^2} \left(r_s - 2e\frac{qQ}{4\pi\epsilon_0}\right)r^3 - \frac{(1-e^2)}{L^2}r^4 \Big] \\&- \frac{1}{r^2} \Big[\frac{r_s}{2} - \left( 1 + \frac{{r_Q}^2 - \left(\frac{qQ}{4\pi\epsilon_0}\right)^2}{L^2} \right) r \\&+ \frac{3}{2 L^2} \left(r_s - 2e\frac{qQ}{4\pi\epsilon_0}\right)r^2 - \frac{2(1-e^2)}{L^2}r^3\Big]\\
    &= -2r_Q^2 u^3 + \frac{3}{2} r_s u^2 -\left( 1 + \frac{{r_Q}^2 - \left(\frac{qQ}{4\pi\epsilon_0 r}\right)^2}{L^2} \right) u \\&+ \frac{1}{2 L^2}\left(r_s - 2e\frac{qQ}{4\pi\epsilon_0}\right)
\end{split}
\end{equation}

Finalmente, haciendo los mismos pasos para la Ec. (\ref{eqlightlike}), obtenemos:
\begin{equation}
\frac{d^2 u}{d \phi^2} = -u + \frac{3}{2} r_s u^2 - 2 {r_Q}^2 u^3
\end{equation}

Ambos resultados se pueden combinar en uno solo, usando la constante $e_0$ de unas secciones atrás. Para la luz $e_0=0$, y para partículas con masa $e_0=1$. El resultado condensado es:
\begin{fullwidth}
\begin{remarkbox}{Ec. diferencial de la órbita en el espacio-tiempo de R-N.}
\begin{equation}
    \frac{d^2 u}{d \phi^2} =\frac{e_0}{2 L^2}\left(r_s - 2e\frac{qQ}{4\pi\epsilon_0}\right) -\left[ 1 + e_0\left( \frac{{r_Q}^2 - \left(\frac{qQ}{4\pi\epsilon_0 r}\right)^2}{L^2} \right)\right] u +\frac{3}{2} r_s u^2 - 2r_Q^2 u^3 
\label{lasube}
\end{equation}
\end{remarkbox}
\end{fullwidth}

\section{La simulación}
\begin{flushright}
\textit{This is your last chance. After this, there is no turning back.\\You take the blue pill — the story ends, you wake up in your bed and believe whatever you want to believe.\\You take the red pill — you stay in Wonderland and I show you how deep the rabbit-hole goes.\\\textbf{Morpheus}}
\end{flushright}

Trabajamos con los siguientes datos:

\begin{itemize}
    \item Masa $M$: 1 Masa solar = $2x10^{30}\ Kg$
    \item Carga $Q$: 0 C
    \item Carga $q$: 0 C
    \item $\theta_{inicial} = \frac{\pi}{2}$ y $\Dot{\theta}_{inicial}=0$.
\end{itemize}

\subsection*{\textbf{Órbita circular de la luz}}
Partiendo de la Ec. (\ref{eqradial3.2}) y de la métrica, junto a la Ecuación de la Órbita (\ref{lasube}), tomando $dr/d\lambda =d^2r/d\lambda^2 = 0$ para una trayectoria \textit{lightlike}, $d\tau^2=0$, se obtiene que para las posibles órbitas circulares de un fotón, la distancia radial tiene que ser:\footnote{Tras un par de pasos algebraicos.} 

\begin{equation}
    r_\pm = \frac{3 r_s}{4} \pm \sqrt{\left(\frac{3 r_s}{4}\right)^2 - 2 {r_q}^2}
\end{equation}

De estos dos radios el único físico es $r_+$, pues el otro siempre es menor al horizonte de eventos. Simulando con las condiciones iniciales:

\begin{equation}
    r_{inicial}=r_+\ \text{, }\ \Vec{c}= 0\Vec{e}_r + c\Vec{e}_{\phi}
\end{equation}

\begin{figure}[h!]
    \centering
    \includegraphics[width=.8\textwidth]{Im/luz_circular.png}
    \caption{Órbita circular}
\end{figure}

\begin{figure}[h!]
    \centering
    \includegraphics[width=.8\textwidth]{Im/luz_interior_angulo91.png}
    \caption{Desviación de 1° la órbita circular hacia el interior}
\end{figure}

\begin{figure}[h!]
    \centering
    \includegraphics[width=.8\textwidth]{Im/luz_exterior_angulo89_1vuelta.png}
    \caption{Desviación de 1° de la órbita circular hacia el exterior}
\end{figure}

\newpage
\subsection*{\textbf{Órbita de Mercurio}}
Uno de los grandes problemas que tuvo la Mecánica Newtoniana es explicar las observaciones respecto a la precesión del perihelio de Mercurio: La teoría predecía una precesión de $531.9\ arc seg/siglo$, mientras que las observaciones medían $575\ arc seg/siglo$ ¡Una diferencia de $43\ arc seg/siglo$! Pasaron décadas hasta que al fin la teoría de la Relatividad General de Einstein mostró que al considerar la curvatura del espacio-tiempo debido al sol, Mercurio precesa aproximadamente $43\ arc seg/siglo$. Y, añadiéndole la contribución de los otros planetas del sistema solar se termina de explicar las observaciones. Simulamos una partícula partiendo del perihelio de mercurio, moviéndose con velocidad radial nula y velocidad tangencial máxima. Las condiciones iniciales\footnote{https://nssdc.gsfc.nasa.gov/planetary/
factsheet/mercuryfact.html} son:
\begin{equation}
    r_{inicial}=46.002x10^6 Km\ \text{, }\ \Vec{V}= 0\Vec{e}_r + (58.98 m/s)\Vec{e}_{\phi}
\end{equation}

\begin{figure}[h!]
    \centering
    \includegraphics[width=.8\textwidth]{Im/rvsphi_mercurio.png}
    \caption{Gráfica de la órbita de Mercurio: $r$ vs $\phi$}
\end{figure}

Midiendo a que ángulos se dan los mínimos, los perihelios, y viendo la diferencia de esos ángulos se puede estimar la precesión del perihelio. Con la precisión que tenemos, este número nos varía entre la mitad del valor real, y el doble del valor real. 

\begin{figure}[h!]
    \centering
    \includegraphics[width=.8\textwidth]{Im/drvsphi_mercurio.png}
    \caption{Gráfica de $\frac{dr}{d\phi}$ vs $\phi$}
\end{figure}

\subsection*{Muchas posibilidades}
Bajo ciertas limitaciones computacionales, es posible jugar con cualquier condición inicial. Dejamos una genial:

\begin{figure}[h!]
    \centering
    \includegraphics[width=.8\textwidth]{Im/Roseta.png}
    \caption{Órbita divertida en forma de flor.}
\end{figure}
\newpage


%Simulations
%\chapter*{\textcolor{myred}{Epílogo}}

\textcolor{black}{\hrule}
\vspace{0.5cm}
{\huge{Conclusiones y Fronteras}}
\vspace{0.5cm}
\textcolor{black}{\hrule}

\newthought{Luego de haber realizado todo este recorrido por agujeros negros y espacio-tiempos curvados}, podemos afirmar que hemos aprendido mucho sobre Relatividad General. Ya sabemos interpretar muchas ecuaciones y métricas, e incluso hallar nuestras propias métricas para sistemas nuevos.

\vspace{0.5cm}

Uno de los resultados más interesantes a los que llegamos, que contradice bastante lo que estudiamos en Física III y Electromagnetismo I y II, es que aquí \textbf{los objetos con carga modifican las trayectorias de cualquier tipo de partícula}, independientemente de si esta última tiene carga o no. Esto es porque la fuente de la curvatura del espacio-tiempo es la \textbf{energía} (tensor de estrés-energía). Que un objeto sin carga pueda interactuar con un objeto con carga es un resultado análogo (e igual de sorprendente) a cuando un objeto sin masa -la \textbf{luz}- interactúa con un objeto con masa.

\vspace{0.5cm}

También fuimos capaces (no sin dificultad) de desarrollar un programa que realiza simulaciones de las trayectorias de partículas en un espacio-tiempo de Reisser-Nordström. Como esto es una generalización de espacio-tiempo de Schwarzschild, el programa también sirve para simular los problemas más conocidos de la Relatividad General: precesión del perihelio de Mercurio, deflección de la luz cerca de un objeto masivo. Incluso pudimos hacer una simulación del Sistema Solar.

\vspace{0.5cm}

Es posible expandir este trabajo realizando un estudio más minucioso del espacio-tiempo cerca del agujero negro de Reissner-Nordström, la presencia de carga eléctrica nos habilita a definir un nuevo radio, denominado \textbf{Radio de Cauchy}, que establece una región nueva entre dicho radio y el horizonte de eventos. Es posible analizar las diferencias entre cada una de las regiones del agujero negro. Este nuevo radio también nos permite postular la existencia de \textbf{singularidades desnudas}, que no tienen horizonte de eventos. Es decir, son agujeros 'negros' \textit{que dejan pasar la luz}.

\vspace{0.5cm}

Los agujeros negros con carga no existen en la realidad: para que los fenómenos aquí estudiados sean apreciables, necesitaríamos una carga del orden de los $10^{18}$ Coulomb. Sin embargo, el interés en este tipo de problemas no es puramente teórico: la métrica de Reissner-Nordström es el primer paso hacia la construcción de una métrica más general, que estudia los agujeros negros que rotan. Esta es la \textbf{métrica de Kerr}. Obviamente, una posible continuación del trabajo aquí realizado debería incorporar los agujeros negros de Kerr (y de Kerr-Newman, que rotan \textit{y} tienen carga eléctrica).

\vspace{0.5cm}

Por último, existe una rama de la Relatividad General llamada \textbf{gravitational lensing}, que estudia cómo la presencia de objetos masivos, al desviar la luz proveniente de estrellas lejanas, modifica las imágenes que pueden observarse desde la Tierra. Esto nos invita a utilizar técnicas computacionales de raytracing para simular cómo se vería un agujero negro.

\begin{figure}
    \centering
    \includegraphics[width=\textwidth]{Im/F1000033-01.jpeg}
    \caption{Amigues relativistas.}
    \label{fig:my_label}
\end{figure}

\newpage
\textcolor{black}{\hrule}
\vspace{0.5cm}
{\huge{Material de Estudio}}
\vspace{0.5cm}
\textcolor{black}{\hrule}


\newthought{A continuación}, vamos a realizar un breve recorrido por los distintos recursos que utilizamos para estudiar estos temas.

\subsection*{\textbf{A General Relativity Workbook}, de Thomas A. Moore}
Este fue nuestro libro de cabecera, y tal como su nombre lo indica, es un libro diseñado para trabajar. Los capítulos tienen entre 5 y 10 páginas de extensión, y la mitad de los temas que desarrolla son propuestas para que resuelva lx estudiante. Abarca todos los temas relevantes para un primer curso de Relatividad General, y utiliza conceptos claros y sintéticos. 
\begin{marginfigure}
\includegraphics[width=0.8\textwidth]{Im/moore.jpg}
\end{marginfigure}

El libro no deja ningún término matemático sin definir, lo cual es ideal para quienes necesitan ver las cuentas para terminar de asimilar un tema (como nosotrxs). Es muy útil como guía conductora, aunque a veces no indaga tanto en lo conceptual para \textit{sacarle el jugo} a los temas.

\subsection*{\textbf{Exploring Black Holes: An Introdution to General Relativity}, de John A. Wheeler y Edwin F. Taylor}

Si en libro anterior tenía mucha matemática y poca interpretación, este libro es todo lo contrario. Sin mencionar tensores ni diferenciación covariante en ningún momento, los autores proponen trabajar con métricas 'dadas' y estudiar al máximo (\textit{explorar}) cómo es la geometría del espacio-tiempo. 
\begin{marginfigure}
\includegraphics[width=0.8\textwidth]{Im/wheeler.jpg}
\end{marginfigure}
De a ratos puede parecer que todo está un poco \textit{en el aire}, en especial para quienes, al igual que nosotrxs, prefieren libros un poco más matemáticos. 

Algo muy interesante es la cantidad de proyectos que propone el libro. De hecho, la mitad del libro son instrucciones muy detalladas para resolver dichos proyectos. Fue el primer libro que encontramos sobre RG, porque justamente estábamos buscando proyectos guiados para resolver.

\subsection*{\textbf{The Classical Theory of Fields}, de Landau y Lifshitz}


Este libro no requiere demasiada introducción: para algunas personas es un \textit{ladrillo}, mientras que para otras es una \textit{caricia al alma}. 
\begin{marginfigure}
\includegraphics[width=0.8\textwidth]{Im/landau.jpg}
\end{marginfigure}
Lo cierto es que es un libro ideal para agarrar una vez que ya manejamos un poco los temas (lo mismo sucede con todos los libros de la serie de Landau), ya que no se priva de nada: tiene todo el formalismo matemático correspondiente a RG \textit{y} es conceptualmente profundo.
Sin dudas es un buen libro para consultar dudas, y en los primeros capítulos nos sirvió para tomar un hilo conductor.

\newpage

\subsection*{\textbf{Clases de Stanford} de Leonard Susskind}

Esta serie de 10 clases es ideal para ver los alcances que tiene la RG, explicados por una persona que lleva muchos años en el tema.
\begin{marginfigure}
\includegraphics[width=1.3\textwidth]{Im/susskind.png}
\end{marginfigure}
Creo que se pueden apreciar mejor después de haber leído los capítulos 3 a 6 y 17 a 19 del Moore, ya que Susskind casi no escribe en el pizarrón y a veces es difícil seguir los conceptos matemáticos más abstractos sin una buena representación visual. Es ideal para estudiar qué sucede en un agujero negro de Schwarzschild y para introducirse en los Diagramas de Penrose.

\subsection*{\textbf{Einstein Field Equations For Beginners!} de DrPhysicsA}

Este video es la manera perfecta de spoilearse RG desmenuzando las Ecuaciones de Einstein. Todo el video está hecho en una hoja de papel y narrado por una voz con un acento muy inglés. 
\begin{marginfigure}
\includegraphics[width=1.3\textwidth]{Im/bumpyy.png}
\end{marginfigure}
Creo que es una buena manera de hacer un pantallazo general de los temas, ya que es muy fácil de entender y nos presenta todos los elementos que vamos a tener que utilizar más adelante.

\subsection*{\textbf{Your Daily Equation} de Brian Greene}

A partir del episodio 26, el autor de \textbf{The Elegant Universe} comienza a explicarnos las Ecuaciones de Campo de Einstein y todos los elementos que la constituyen.
\begin{marginfigure}
\includegraphics[width=1.3\textwidth]{Im/bumpy.png}
\end{marginfigure}
Da muchos ejemplos para hacer entender los conceptos, y de ahí sacamos gran parte de la sección \textit{GR In a Nutshell} de este apunte.

\subsection*{\textbf{The Maths of General Relativity} de Science Clic}

Esta es una serie de 8 videos que empezó a subirse a finales de 2020 y fue terminada en enero de 2021. Las animaciones están muy bien hechas y permiten entender varios conceptos de manera rápida y clara. 

Se complementa muy bien con libros como el Moore o el Landau, ya que en una animación de 15 segundos sintetiza lo que Landau puede estar dos páginas enteras explicando sin usar una sola imagen.
\begin{marginfigure}
\includegraphics[width=1.3\textwidth]{Im/scienceclic.png}
\end{marginfigure}

\begin{marginfigure}
\includegraphics[width=0.8\textwidth]{Im/qr_img.png}
\end{marginfigure}
\vspace{4cm}
\newthought{En el siguiente link}, encontrarán toda información relacionada con este trabajo, así como también los códigos que hicimos para realizar las simulaciones y varios videos de dichas simulaciones.

\textcolor{myred}{\underline{\url{https://drive.google.com/drive/folders/1AZ_T-1sP442UiVBRkR45JsUwid0Mn6Qu?usp=sharing}}}





\end{document}