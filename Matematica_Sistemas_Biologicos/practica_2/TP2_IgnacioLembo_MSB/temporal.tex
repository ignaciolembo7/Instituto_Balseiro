\chapter{\textcolor{myred}{La métrica de Reissner-Nordström}}

%Principio de equivalencia (fuerte): En un laboratorio en caída libre (no rotante) que ocupa una pequeña región del espacio tiempo, las leyes de la física son las correspondientes a la relatividad especial
\subsection*{\textbf{Sobre las unidades geométricas}}
A partir de ahora, salvo que indiquemos lo contrario, usaremos las unidades geométricas. Esto es, tomaremos la velocidad de la luz $c$, y la constante de gravitación universal $G$ como :

\begin{remarkbox}{Consideración de las unidades geométricas}
\begin{equation*}
    c=G=1\ \ \textit{adimensional}
\end{equation*}
\end{remarkbox}

Hacemos esto para olvidarnos de las constantes, y hacer menos engorrosas las ecuaciones.
\begin{table}[h]
  \begin{center}
    \begin{tabular}{lccl}
      \toprule
      Variable & Unidades SI & Unidades Geom. & Factor \\
      \midrule
      Masa & $kg$ & $m$ &$c^2 G^{-1}$ \\
      Longitud & $m$ & $m$ & 1 \\
      Tiempo & $s$ & $m$ & $c^{-1}$\\
      Velocidad & $m s^{-1}$ & adim & $c$  \\
      Aceleración & $m s^{-2}$ & m{-1} & $c^2$  \\
      Fuerza & $kg m s^{-2}$ & adim & $c^4 G^{-1}$  \\
      Momento Angular & $kg m^2 s^{-1}$ & $m^2$ & $c^3 G^{-1}$ \\
      Momento & $kg m s^{-1}$ & $m$ & $c^3 G^{-1}$ \\
      Energía & $kg m^2 s^{-2}$ & $m$ & $c^{4} G^{-1}$\\
      Densidad de Energía & $kg m^{-1} s^{-2}$ & $m^{-2}$ & $c^4 G^{-1}$  \\
      \bottomrule
    \end{tabular}
  \end{center}
  \caption{Unidades Geométricas. Para convertir Geom. $\rightarrow$ SI, multiplicar por el factor. Para convertir SI $\rightarrow$ Geom., dividir por el factor. De forma general, para unidades SI de "$kg^\alpha m^\beta s^\gamma$", las unidades geométricas son "$m^{\alpha + \beta + \gamma}$".}
  \label{geounits}
\end{table}
\section{\huge{Geodésicas en la Geometría de Reissner-Nordström}}

\textcolor{myred}{\hrule}

\newthought{Ahora encontremos las ecuaciones que describen el movimiento de fotones y partículas no cargadas.} La partícula seguirá una geodésica \textbf{\textit{time-like}} mientras que el fotón una geodésica \textbf{\textit{light-like}}. Sea $x^\alpha = x^\alpha (\lambda)$ una curva parametrizada por $\lambda$, entonces debe cumplir la Ec. (\ref{geodesicx}):
\begin{equation}
    \frac{d^2 x^\alpha}{d \lambda^2} + \Gamma^\alpha_{\mu\nu} \frac{d x^\mu}{d \lambda} \frac{d x^\nu}{d \lambda} = 0
\end{equation}

Donde $\Gamma^\alpha_{\mu\nu}$ son los símbolos de Christoffel asociados a la métrica. Para una geodésica \textit{time-like} lo más natural es definir el parámetro $\lambda$ como el tiempo propio $\tau$, mientras que una geodésica \textit{light-like} no puede parametrizarse con $\tau$. Mantendremos por ahora el parámetro $\lambda$ para trabajar con los dos casos. Reemplazando los símbolos de Christoffel para nuestra métrica las ecuaciones de movimiento son:
 
\newthought{Para $\alpha=0$:}
\begin{equation}
    \frac{d^2 t}{d \lambda^2} + \frac{A^\prime}{A} \frac{d t}{d \lambda} \frac{d r}{d \lambda} = 0
\label{eqt3.1}
\end{equation}
\newthought{Para $\alpha=1$:}
\begin{equation}
\begin{split}
    \frac{d^2 r}{d \lambda^2} + \frac{A^\prime}{2B} \left(\frac{d t}{d \lambda}\right)^2 + \frac{B^\prime}{2B} \left(\frac{d r}{d \lambda}\right)^2 - \frac{r}{B} \left(\frac{d \theta}{d \lambda}\right)^2 &\\- \frac{r \sin^2{\theta}}{B} \left(\frac{d \phi}{d \lambda}\right)^2 &= 0
\label{eqradial3.1}
\end{split}
\end{equation}
\newthought{Para $\alpha=2$:}
\begin{equation}
    \frac{d^2 \theta}{d \lambda^2} + \frac{2}{r} \frac{d \theta}{d \lambda} \frac{d r}{d \lambda} - \sin{\theta}\cos{\theta} \left(\frac{d \phi}{d \lambda}\right)^2= 0
\end{equation}
\newthought{Para $\alpha=3$:}
\begin{equation}
    \frac{d^2 \phi}{d \lambda^2} + \frac{2}{r} \frac{d \phi}{d \lambda} \frac{d r}{d \lambda} - 2\cot{\theta}\frac{d \phi}{d \lambda} \frac{d \theta}{d \lambda}= 0
\label{eqazimutal3.1}
\end{equation}

Debido a la simetría esférica la trayectoria debe estar contenida en un plano definido por las condiciones iniciales\footnote{Trazamos el plano que contiene la velocidad en el instante inicial, ya sea de la partícula o del fotón. Sí la trayectoria se saliera de ese plano, la dirección en la que lo haga sería preferencial, rompiendo la simetría esférica del sistema.}, por lo que podemos poner sin ninguna pérdida de generalidad que $\theta=\pi/2$ en todo momento. Esto nos anula las derivadas de $\theta$, y las ecuaciones de movimiento se simplifican. Ahora, las Ecs. (\ref{eqradial3.1}) y (\ref{eqazimutal3.1}) nos quedan:

\begin{equation}
    \frac{d^2 r}{d \lambda^2} + \frac{A^\prime}{2B} \left(\frac{d t}{d \lambda}\right)^2 + \frac{B^\prime}{2B} \left(\frac{d r}{d \lambda}\right)^2 -  \frac{r}{B} \left(\frac{d \phi}{d \lambda}\right)^2 = 0
\label{eqradial3.2}
\end{equation}
\begin{equation}
    \frac{d^2 \phi}{d \lambda^2} + \frac{2}{r} \frac{d \phi}{d \lambda} \frac{d r}{d \lambda}= 0
\label{eqazimutal3.2}
\end{equation}

Sí dividimos la Ec. (\ref{eqazimutal3.2}) por $d\phi/d\lambda$, y usamos que:
\begin{equation}
\begin{split}
    \left(\frac{d\phi}{d\lambda}\right)^{-1} \frac{d^2\phi}{d\lambda^2} &= \frac{d}{d\lambda} \log{\left(\frac{d\phi}{d\lambda}\right)}\\
    \text{y, }\ \frac{2}{r} \frac{dr}{d\lambda} &= \frac{d}{d\lambda}\log{(r^2)}
\end{split}
\end{equation}
llegamos a que:

\begin{equation}
    \frac{d}{d\lambda}\log{\left(r^2\frac{d\phi}{d\lambda}\right)}=0 \Rightarrow r^2 \frac{d\phi}{d\lambda} = L = cte
\label{eqL}
\end{equation}

Donde $L$ es una constante del movimiento que coincide con el \textit{momento angular por unidad de masa} de la teoría Newtoniana. De manera similar, obtenemos para la Ec. (\ref{eqt3.1}) que:

\begin{equation}
    \frac{d}{d\lambda}\log{\left(A\frac{dt}{d\lambda}\right)}=0 \Rightarrow A \frac{dt}{d\lambda} = e = cte
\label{eqe}
\end{equation}

Con $e$ una constante del movimiento que se puede interpretar como la \textit{energía total relativista por unidad de masa}. Ahora usamos las Ecs. (\ref{eqL}) y (\ref{eqe}) en la Ec. (\ref{eqradial3.2}) y tenemos:

\begin{equation}
    \frac{d^2 r}{d \lambda^2} + \frac{A^\prime}{2B} \frac{e^2}{A^2} + \frac{B^\prime}{2B} \left(\frac{d r}{d \lambda}\right)^2 - \frac{r}{B} \frac{L^2}{r^4} = 0
\end{equation}
recordando que $B\prime=-A\prime/A^2$ y multiplicando por $2Bdr/d\lambda$:
\begin{equation}
\begin{split}
    0&=2B\frac{d^2 r}{d \lambda^2}\frac{dr}{d\lambda} - e^2 B^\prime \frac{dr}{d\lambda} + B^\prime \frac{dr}{d\lambda} \left(\frac{d r}{d \lambda}\right)^2 - \frac{2L^2}{r^3} \frac{dr}{d\lambda}\\
    &=\frac{d}{d\lambda}\left[ B \left(\frac{d r}{d \lambda}\right)^2 - e^2 B + \frac{L^2}{r^2}\right]
\end{split}
\end{equation}
por lo que:
\begin{equation}
    B \left(\frac{d r}{d \lambda}\right)^2 - e^2 B + \frac{L^2}{r^2} = -(e_0)^2 = cte
\label{eqe0}
\end{equation}

Donde $e_0$ se puede pensar como la \textit{energía total en reposo por unidad de masa}. Reescribiendo la Ec. (\ref{eqe0}) obtenemos:
\begin{equation}
    \left(\frac{d r}{d \lambda}\right)^2 = e^2 -A\left( \frac{L^2}{r^2} + {e_0}^2\right)
\label{eqradial3.3}
\end{equation}

Esta ecuación nos da $dr/d\lambda$ en función de $r$. Para obtener una ecuación de $dr/d\phi$ dividimos la Ec. (\ref{eqradial3.3}) por $(d\phi/d\lambda)^2 = L^2/r^4$:
\begin{equation}
    \left( \frac{dr}{d\phi} \right)^2 = \frac{r^4 e^2}{L^2} - r^2 A \left(1 + \frac{r^2 {e_0}^2}{L^2} \right)
\end{equation}
recordando que $A=1 - (r_s/r) + (r_Q/r)^2$, llegamos a:
\begin{equation}
\begin{split}
    \left( \frac{dr}{d\phi} \right)^2 = -{r_Q}^2 + r_s r - \left(1 + \frac{{r_Q}^2 {e_0}^2}{L^2} \right) r^2 &\\+ \frac{r_s {e_0}^2}{L_2} r^3 &- \frac{{e_0}^2 - e^2}{L^2} r^4
\end{split}
\label{eqrposta}
\end{equation}

Esta es la ecuación que buscamos, nos da $dr/d\phi$ en términos de $r$. En teoría ya con esto podemos encontrar la trayectoria de una partícula o fotón, tomando los valores de las constantes a partir de las condiciones iniciales. Sin embargo, la Ec. (\ref{eqrposta}) se puede simplificar aún más apelando a la métrica\footnote{Recordemos que $c=G=1$ y $\theta=\pi/2$.}:

\begin{equation}
    ds^2 = d\tau^2 = A dt^2 - \frac{1}{A} dr^2 - r^2 d\phi^2
\end{equation}

De las Ecs. (\ref{eqL}), (\ref{eqe}) y (\ref{eq0}) tenemos:
\begin{equation}
\begin{split}
    d\phi^2 &= \frac{L^2}{r^4} d\lambda^2\\
    dt^2 &= \frac{e^2}{A^2} d\lambda^2\\
    dr^2 &= \left[e^2 - A\left({e_0}^2 + \frac{L^2}{r^2}\right)\right] d\lambda^2
\end{split}
\end{equation}

Sustituyendo $d\phi^2$, $dt^2$ y $dr^2$ en la métrica conseguimos:

\begin{remarkbox}{Relación entre $d\tau^2$ y $d\lambda^2$.}
\begin{equation}
    d\tau^2 = {e_0}^2 d\lambda^2
\label{doujou}
\end{equation}
\end{remarkbox}

Finalmente, sí tenemos una partícula con masa y parametrizamos con el tiempo propio, $d\lambda^2=d\tau^2$, según la Ec. (\ref{doujou}) tendremos que $e_0=1$. Sí tenemos un fotón el tiempo propio es nulo, $d\tau^2=0$, y por lo tanto $e_0=0$. Usando este resultado, junto a la Ec. (\ref{eqrposta}) escribimos al fin la ecuación de la trayectoria de un cuerpo no cargado:

\newthought{Para la luz:}
\begin{equation}
    \left( \frac{dr}{d\phi} \right)^2 = -{r_Q}^2 + r_s r - r^2 - \frac{e^2}{L^2} r^4
\label{eqlightlike}
\end{equation}
\newthought{Para un cuerpo, no cargado, con masa:}
\begin{equation}
    \left( \frac{dr}{d\phi} \right)^2 = -{r_Q}^2 + r_s r - \left(1 + \frac{{r_Q}^2}{L^2} \right) r^2 + \frac{r_s}{L_2} r^3 - \frac{1 - e^2}{L^2} r^4
\label{eqtimelike}
\end{equation}

\subsection*{\textbf{Trayectoria de un cuerpo cargado en la Geometría de R-N.}}
\newthought{Teniendo el espacio-tiempo \textit{lleno} de campo electromagnético, es natural preguntarse} la trayectoria de una carga. Para ello nos dirigimos al formalismo de Lagrange. Para una partícula cargada, en un espacio-tiempo cuya curvatura está descrita por $g_{\mu\nu}$, el lagrangiano es\footnote{Sí no hubiera campo electromagnético, $A_\mu=0$ y la solución se correspondería con el lagrangiano de una partícula bajo influencia de un campo gravitatorio. El término correspondiente a la "energía potencial" está incluido en el término con la métrica. En espacio plano, sin campos, se reduce a la energía cinética de una partícula en el vacío.}:

\begin{equation}
    \mathcal{L} = \frac{1}{2} g_{\mu\nu} \dot{x^\mu}\dot{x^\nu} + q A_\mu \dot{x^\mu}
\end{equation}

Donde $q$ es la \textit{carga por unidad de masa} de la partícula, $A_\mu$ es el cuadri-potencial electromagnético y los "puntos" representan una derivada respecto al tiempo propio $\tau$ de la partícula. En nuestro caso solo tenemos campo eléctrico radial, y el cuadri-potencial resulta ser:
  
\begin{equation}
    A_\mu=[A_0,A_1,A_2,A_3]=\left[\frac{Q}{4\pi\epsilon_0 r^2},\ 0,\ 0,\ 0 \right]
\end{equation}

Y, usando la métrica de Reissner-Nordström, el lagrangiano es:

\begin{equation}
    \mathcal{L} = \frac{1}{2}\left( A\dot{t}^2 - \frac{1}{A} \dot{r}^2 - r^2 \dot{\theta}^2 - r^2 \sin^2{\theta} \dot{\phi}^2 \right) + q \frac{Q}{4\pi\epsilon_0 r^2} \dot{t}
\end{equation}

Como antes la trayectoria será en un plano, así que podemos sin pérdida de generalidad considerar $\theta=\pi/2$ y $\dot{\theta}=0$. Luego, las ecuaciones de Euler-Lagrange son:

\newthought{Para $t$:}
\begin{equation}
\frac{d}{d\tau}\frac{\partial \mathcal{L}}{\partial \dot{t}} - \frac{\partial \mathcal{L}}{\partial t} = \frac{d}{d\tau}\left[ A\dot{t} + \frac{qQ}{4\pi\epsilon_0 r^2} \right]  = 0
\label{eqt3.4}
\end{equation}
\newthought{Para $r$:}
\begin{equation}
    \frac{d}{d\tau}\frac{\partial \mathcal{L}}{\partial \dot{r}} - \frac{\partial \mathcal{L}}{\partial r} = -\frac{\ddot{r}}{A} - \frac{A^\prime}{2} \dot{t}^2 - 2 r \dot{\phi}^2 + \frac{qQ}{2\pi\epsilon_0 r^3}  = 0
\label{eqradial3.4}
\end{equation}
\newthought{Para $\phi$:}
\begin{equation}
    \frac{d}{d\tau}\frac{\partial \mathcal{L}}{\partial \dot{\phi}} - \frac{\partial \mathcal{L}}{\partial \phi} =\frac{d}{d \tau}\left[ r^2 \dot{\phi} \right] = 0
\label{eqazimutal3.4}
\end{equation}

De las Ecs. (\ref{eqt3.4}) y (\ref{eqazimutal3.4}) tenemos las constantes de movimiento:

\begin{equation}
    A\dot{t} + \frac{qQ}{4\pi\epsilon_0 r} = e\ \ \text{, y }\ r^2 \dot{\phi} = L
\label{conservposta}
\end{equation}
la única diferencia de las constantes con el caso de la partícula sin carga, es el término $e$ que aparece la energía potencial electrostática $\frac{qQ}{4\pi\epsilon_0 r}$. Ahora, para llegar a una expresión de $dr/d\phi$ como función de $r$, necesitamos apelar a la métrica:

\begin{equation}
    ds^2=d\tau^2=Adt^2 - \frac{1}{A}dr^2- r^2d\phi^2
\end{equation}
Dividimos por $d\tau^2$, y multiplicamos por $A$:
\begin{equation}
    \dot{r}^2 + A - A^2 \dot{t}^2 + A r^2 \dot{\phi}^2 =0
\end{equation}

Usando las Ecs. (\ref{conservposta}), sustituimos $\dot{\phi}$ y $\dot{t}$:
\begin{equation}
    \dot{r}^2 + A - \left(e-\frac{qQ}{4\pi\epsilon_0 r}\right)^2 + A \frac{L^2}{r^2} =0
\end{equation}

Finalmente, dividiendo por $\dot{\phi}$:

\newthought{Para un cuerpo cargado con masa:}
\begin{equation}
\begin{split}
    \left( \frac{dr}{d\phi} \right)^2 &= -r^2 A\left(1+\frac{r^2}{L^2}\right) + \frac{r^4}{L^2}\left( e - \frac{qQ}{4\pi\epsilon_0 r} \right)^2\\
    &=-{r_Q}^2 + r_s r - \left( 1 + \frac{{r_Q}^2 - \left(\frac{qQ}{4\pi\epsilon_0 r}\right)^2}{L^2} \right) r^2 \\ &+ \frac{1}{L^2} \left(r_s - 2e\frac{qQ}{4\pi\epsilon_0 r}\right)r^3 - \frac{(1-e^2)}{L^2}r^4
\end{split}
\end{equation}

Esta es la ecuación que describe la órbita de una partícula cargada en el espacio-tiempo de Reissner-Nordström. Como era esperado, se reduce a la Ec. (\ref{eqtimelike}) cuándo hacemos $q=0$.

%\subsection*{\textbf{Horizonte de Eventos}}

%\newthought{Para encontrar singularidades} tenemos que mirar donde las componentes de la métrica no están definidas. Para la métrica de R-N, la componente crucial es:

%\begin{equation}
%    g_{tt}=-\frac{1}{g_{rr}} = A = 1 - \frac{r_s}{r} + \frac{{r_Q}^2}{r^2}
%\end{equation}

%Los puntos singulares de la métrica son $r=0$ y $r$ tal que $A=1/g_{rr}=0$. Calculemos las raíces de $A$:
%\begin{equation}
%    A=1 - \frac{r_s}{r} + \frac{{r_Q}^2}{r^2}=0 \Rightarrow r^2 - r_s r + (r_Q)^2 = 0
%\end{equation}
%cuyas soluciones son:
%\begin{equation}
%    r_\pm = \frac{1}{2}\left(r_s \pm \sqrt{{r_s}^2 - (2r_Q)^2}\right)
%\label{eqhorizonte}
%\end{equation}

%Entonces tenemos tres puntos singulares: $r=r_+$, $r=r_-$ y $r=0$. Dependiendo de los valores de $r_s$ y $r_Q$ puede que la Ec. (\ref{eqhorizonte}) tenga dos, una o ninguna solución. Esto nos motiva a analizar tres casos distintos:

%\newthought{Para $r_s>2r_Q$} tenemos dos raíces distintas $r_+$ y $r_-$. Llamamos $r_+$ el horizonte de eventos porque para radios menores a ese, ni la luz puede escaparse del mismo. El nombre de "agujero negro" se refiere entonces a un cuerpo cuyas dimensiones sean menores a $r_+$\footnote{Recordemos que todo lo que estudiamos de la métrica de R-N vale solo para una región del espacio vacío. Todo lo que esté "dentro" del cuerpo yace por fuera del alcance de la métrica que desarrollamos.}. Podemos dividir el espacio en tres regiones:
%\begin{itemize}
%    \item Región 1: $r_+<r<\infty$
%    \item Región 2: $r_-<r<\infty$
%    \item Región 3: $0<r<r_-$
%\end{itemize}

%Se puede demostrar que un cuerpo en la región 2 está obligado a disminuir su coordenada $r$, al menos hasta entrar en la región 3 donde eso deja de ser cierto y el cuerpo no está obligado a estrellarse con la singularidad en $r=0$.