\chapter{\textcolor{myred}{La métrica de Reissner-Nordström}}

%Principio de equivalencia (fuerte): En un laboratorio en caída libre (no rotante) que ocupa una pequeña región del espacio tiempo, las leyes de la física son las correspondientes a la relatividad especial

\section{\huge{Resolviendo la ecuación de Einstein}}

\textcolor{myred}{\hrule}

\newthought{En los capítulos} anteriores hemos definido una buena cantidad de conceptos y herramientas correspondientes a la teoría de la Relatividad General. En este capítulo, buscaremos una solución a la ecuación de campo de Einstein-Maxwell para un cuerpo\footnote{Notesé como nos referimos a un cuerpo con ciertas características, no a un agujero negro en particular.} puntual con carga. 

\newthought{El primer paso} para resolver la ecuación de campo de Einstein es definir un buen sistema de coordenadas. Hemos visto en los capítulos anteriores que la geometría del espacio-tiempo alrededor de una distribución de masa y energía está determinada por dicha ecuación. No obstante, tenemos libertad para elegir las coordenadas que utilicemos para describir dicha geometría. Como hacemos siempre en Física, utilizaremos un sistema de coordenadas que explote las simetrías del problema que estemos resolviendo. Luego teniendo en cuenta, estas simetrías podremos proponer una métrica que contenga la menor cantidad de componentes incógnitas posibles.
\begin{marginfigure}
\begin{extrabox}{}
Los pasos para resolver la ecuación se pueden condensar en:
\begin{itemize}
    \item Usar la simetría del problema para definir un sistema de coordenadas lo más completo posible.
    \item Proponer una métrica con la menor cantidad de coeficientes indeterminados posibles.
    \item Sustituir la métrica de prueba en la ecuación de Einstein.
    \item Resolver el sistema de ecuaciones diferenciales para las componentes incógnitas.
\end{itemize}
\end{extrabox}
\end{marginfigure}
Una vez que tengamos esta métrica de prueba la sustituimos en la ecuación de Einstein y obtenemos un sistema de ecuaciones diferenciales acopladas, el cual, si somos capaces de resolverlo, habremos encontrado un sistema de coordenadas que describe la geometría buscada, es decir obtuvimos la métrica para del  espacio-tiempo para el sistema estudiado.

\section{\huge{La forma general de una métrica esféricamente simétrica}}

\textcolor{myred}{\hrule}

\subsection*{\textbf{Consideraciones sobre la forma general de la métrica}}

\newthought{Para poder resolver} las ecuaciones, tendremos que asumir ciertas características de la solución:

\vspace{0.75cm}

\begin{remarkbox}{Consideraciones sobre la métrica de Reissner-Nordström}
\begin{itemize}
    \item Consideramos el espacio tiempo alrededor de una fuente esféricamente simétrica.
    \item El espacio es vacío excepto por la presencia de campos electromagnéticos.
    \item Cuando la carga del cuerpo tiende a 0 $(Q \to 0)$ la métrica debe ser la de Schwarzchild.
    \item Cuando la distancia al objeto tiende a infinito $(r \to \infty)$ la métrica debe acercarse a la de Minkowski.
\end{itemize}
\end{remarkbox}

\subsection*{\textbf{Elección de las coordenadas espaciales}}

Simetría esférica significa que podemos definir un conjunto anidado de superficies bidimensionales concéntricas en el espacio-tiempo alrededor de la fuente cuya geometría intrínseca es la misma que la de una esfera ordinaria en dos dimensiones. Si definimos las coordenadas angulares $\theta$ (ángulo polar) y $\phi$ (ángulo azimutal) en la forma usual para cada superficie, y definimos $r$ como la circunferencia de tal esfera dividida por $2\pi$. Luego la métrica para cada superficie esférica es 
\begin{equation}
    ds^2 = r^2(d\theta^2+\sin^2{\theta}d\phi^2)
    \label{metricaRN1}
\end{equation}

Esta ''parte'' de la métrica solo aplica para una de las esferas anidadas ($r$ constante), es decir tenemos las componentes del tensor métrico que tienen que ver con las coordenadas angulares. Hasta ahora no hay nada que nos prohiba darle a cada esfera su propio set de coordenadas angulares. Pero podemos alinear los sistemas coordenados de todas las esferas concéntricas, pidiendo que la linea curva definida por $\theta = cte$ y $\phi=cte$ sean perpendiculares a cada esfera. Esto significa que los vectores $\bm{e}_\theta$ y $\bm{e}_\phi$ sean perpendiculares a $\bm{e}_r$. Recordando \ref{tensormetricobases} lo anterior nos dice que
\begin{equation*}
    g_{r\theta} = \bm{e}_r  \cdot \bm{e}_\theta = 0 \hspace{1cm} g_{r\phi} = \bm{e}_r  \cdot \bm{e}_\phi =0
\end{equation*}
Entonces, tenemos las componentes del tensor métrico que tienen que ver con $r$, donde $g_{rr}$ es incógnita por ahora y podemos escribir la parte puramente espacial de la métrica
\begin{equation}
    ds^2 = g_{rr}dr^2 + r^2(d\theta^2+\sin^2{\theta}d\phi^2)
    \label{metricaRN2}
\end{equation}

Ahora, supongamos que tenemos un término de la métrica que sea de la forma $g_{t\phi}dtd\phi$, esto implicaría que la geometría del espacio tiempo trata a los desplazamientos dondo $d\phi > 0$ distinto a los desplazamientos donde $d\phi <0$. Esto daría una dirección preferencial al movimiento contrario a la suposición de simetría esférica\footnote{La métrica de Kerr para agujeros negros rotantes, donde no hay simetría esférica esta componente del tensor no es nula, lo cual complica considerablemente las cosas.}. Entonces, la simetría esférica del sistema nos permite elegir usar coordenadas de modo tal que $g_{t\phi}=0$. Por un argumento, similar $g_{t\theta}=0$. Finalmente la métrica que tiene en cuenta el espacio y el tiempo tiene la forma 
\begin{equation}
    ds^2 = g_{tt}dt^2 + 2g_{rt}drdt +g_{rr}dr^2 + r^2(d\theta^2+\sin^2{\theta}d\phi^2)
\end{equation}
donde escribimos $2g_{rt}drdt$ por la simetría del tensor métrico. 

\subsection*{\textbf{Elección de la coordenada temporal}}

Hasta ahora la coordenada temporal es completamente arbitraria. Podemos usar nuestra libertad en la elección de las coordenadas para definir $g_{tt}$ y $g_{rt}$, pero teniendo cuidado que hay ciertas restricciones. Primero, $g_{tt} \neq 0$ o si no, no tendríamos espacio-\textit{tiempo}. Segundo, debe ser negativo cuando las otras componentes de la métrica son positivos y $g_{tr}=0$ (para que exista el caso lightlike $ds^2=0$). Supongamos que realizamos alguna elección arbitraria de las coordenadas temporales que definen $g_{tt}$ y utilizamos las ecuaciones de Einstein para encontrar $g_{rt}$. Si realizamos una transformación de coordenadas de la forma 
\begin{equation*}
    t' = t + f(r,t)
\end{equation*}
se puede demostrar que una elección apropiada de dicha función setea $g'_{rt}=0$ en nuestro sistema de coordenadas transformado\footnote{El ' significa en el sistema transformado, no derivada con respecto a algo.}. Pero si podemos elegir una coordenada temporal de modo tal que se cumpla $g'_{rt}=0$, por la arbritrariedad en la elección podemos elegir directamente $g_{rt}=0$ antes de resolver la ecuación de Einstein.\footnote{Eliminar elementos de la diagonal representa una gran ventaja a la hora de realizar los cálculos.}.

Finalmente, una forma general para la métrica esféricamente simétrica es
\begin{remarkbox}{Forma general para una métrica esféricamente simétrica }
\begin{equation*}
    ds^2 = A(r,t)dt^2 - B(r,t)dr^2 - r^2d\theta^2 -r^2\sin^2{\theta}d\phi^2
\end{equation*}
\end{remarkbox}

donde elegimos $\phi$ y $\theta$ como las coordenadas angulares azimutal y polares de siempre, $r$ como la coordenada radial circunferencial, y $t$ de modo tal que $g_{rt}$.

\section{\huge{Reemplazando la forma general de la métrica en la ecuación de Einstein}}

\newthought{Siguiendo} la 'receta' que presentamos antes, ahora debemos utilizar la forma de la métrica que hallamos y reemplazarla directamente en la ecuación de campo de Einstein que recordemos está dada por (\ref{campoeinsteincontravariante}), tomando la constante cosmológica como $0$ tenemos\footnote{Hemos agregado un término $c^{-4}$ que antes no estaba por simplicidad de las cuentas pero ahora nos será útil trabajar en el Sistema Internacional de Unidades (SI).}
\begin{equation*}
R^{\mu\nu}= \frac{8 \pi G}{c^4}\left(T^{\mu\nu}-\frac{1}{2} g^{\mu\nu} T\right)
\end{equation*}
Además, recordando que estamos trabajando en el vacío (solo con presencia de campos electromagnéticos), el tensor de energía-momento está dado por (\ref{tensorestresenergiafunciondeF})
\begin{equation*}
T^{\mu\nu}=\frac{1}{4\pi k}\left(\frac{1}{4}g^{\mu\nu}F^{\sigma\gamma}F_{\sigma\gamma}-F^{\mu\alpha}g_{\alpha\beta}F^{\nu\beta}\right)
\end{equation*}
y para este tensor calculamos que su traza era nula (ec. \ref{traceless}), por lo que la ecuación de campo de Einstein que usaremos es
\begin{equation}
R^{\mu\nu}=\frac{8 \pi G}{c^4}T^{\mu\nu}
\label{riccicosmonula}
\end{equation}
\subsection*{\textbf{Bajar los índices 'upstairs' a 'downstairs'}}

En esta sección expresaremos el tensor de estrés-energía (ec. \ref{tensorestresenergiafunciondeF} y el tensor de Ricci (ec. \ref{riccicosmonula}) de forma covariante (con los índices abajo \textit{downstairs}), dado que nos será útil más adelante. 

Para bajar dichos índices multiplicamos a ambos lados por el tensor métrico\begin{marginfigure}
\begin{extrabox}{}
Para bajar o subir índices una regla que funciona es pensar que el segundo índice del tensor métrico índica qué índice del tensor sube o baja (es el que está repetido) y el primer índice del tensor métrico indica por qué índice se remplaza. Por ejemplo:
\begin{equation}
    g^{\alpha\gamma} R_{\alpha\beta\gamma\sigma} = {{R_{\alpha\beta}}^\alpha}_\sigma
\end{equation}
\end{extrabox}
\end{marginfigure} de forma conveniente entonces para el tensor de Ricci por ejemplo:
\begin{equation}
\begin{split}
g_{\alpha\mu}g_{\beta\nu}R^{\mu\nu} &= \frac{8 \pi G}{c^4}g_{\alpha\mu}g{\beta\nu}T^{\mu\nu} \\    
g_{\alpha\mu}R^{\mu}_\beta &= \frac{8 \pi G}{c^4}g_{\alpha\mu}T^{\mu}_\beta \\    
R_{\alpha\beta} &= \frac{8 \pi G}{c^4}T_{\alpha\beta}   
\end{split}
\label{riccidownstairs}
\end{equation}
De manera similar podemos bajar los indices del tensor de estrés-energía
\begin{equation}
T_{\alpha\beta}=\frac{1}{\mu_0}\left(\frac{1}{4}g_{\alpha\beta}F_{\mu\nu}F^{\mu\nu}-g_{\beta\nu}F_{\alpha\mu}F^{\nu\mu}\right)
\label{tensorestresdownstairs}
\end{equation}

\subsection*{\textbf{Cálculo del tensor de Ricci}}

Comencemos por el lado izquierdo de la ecuación de Einstein en forma covariante (\ref{riccidownstairs}), es decir por el cálculo del tensor de Ricci. Para calcular las componentes de dicho tensor debemos primero calcular los símbolos de Christoffel a partir de la ecuación (\ref{simboloChristoffel}) 
\begin{equation*}
\Gamma^{\alpha}_{\mu \nu}=\frac{1}{2}g^{\alpha\sigma}\left[\partial_{\mu}g_{\nu\sigma}+\partial_{\nu}g_{\sigma \mu}-\partial_{\sigma}g_{\mu\nu}\right]
\end{equation*}

Por lo que necesitamos el tensor métrico en su forma covariante y contravariante. Hasta ahora, a partir de la métrica que tenemos $g_{\mu\nu}$ y $g^{\mu\nu}$ son\footnote{Ya vimos que en la representación matricial son matrices inversas.} 
\begin{equation}
g_{\mu\nu} = \begin{pmatrix}
Ac^2 & 0 & 0 & 0 \\
0 & -B & 0 & 0 \\
0 & 0 & -r^2 & 0 \\
0 & 0 & 0 & -r^2\sen^2{\theta}
\end{pmatrix}
\hspace{0.2cm}
g^{\mu\nu} = \begin{pmatrix}
\frac{1}{Ac^2} & 0 & 0 & 0 \\
0 & \frac{-1}{B}  & 0 & 0 \\
0 & 0 & \frac{-1}{r^2}  & 0 \\
0 & 0 & 0 & \frac{-1}{r^2\sen^2{\theta}} 
\end{pmatrix}
\label{tensoresmetricos}
\end{equation}

Veamos que para calcular los símbolos de Christoffel solo debemos calcular las derivadas parciales de las entradas del tensor métrico y ser muy prolijos con los índices. Esta tarea no es para nada difícil, pero sí muy larga y tediosa. A modo de ejemplo, mostramos en el apéndice ... el cálculo del símbolo ..... Esta tarea por suerte puede ser simplificada mediante métodos computacionales\cite[][p. 26]{Jhonny} y los símbolos de Christoffel no nulos resultan\footnote{Existen varios softwares que permiten el cálculo de simbolos de Christoffel, tensores de Ricci, de Riemann. Maple 17 es un software que permite realizar este tipo de cálculo para entradas incógnitas como los A y B que tenemos, lamentablemente es privado. Einstein.py es un paquete de Python de carácter libre que también permite realizar este tipo de cálculos, pero (hasta donde sabemos) solo para métricas ya dadas. Es decir, no podemos realizar derivadas simbólicas para los $A$ y $B$ que necesitamos.} 

\begin{equation}
\begin{split}
&\Gamma^{0}_{00} = \frac{\dot{A}}{2Ac} \hspace{1cm} \Gamma^{1}_{01} = \Gamma^{1}_{10} = \frac{\dot{B}}{2Bc} \\
&\Gamma^{0}_{11} = \frac{\dot{B}}{2Ac} \hspace{1cm} \Gamma^{0}_{01} = \Gamma^{0}_{10} = \frac{A'}{2A} \\
&\Gamma^{1}_{00} = \frac{A'}{2B} \hspace{1cm} ~~\Gamma^{2}_{12} = \Gamma^{2}_{21} = \frac{1}{r} \\
&\Gamma^{1}_{11} = \frac{B'}{2B} \hspace{1cm} ~~\Gamma^{3}_{13} = \Gamma^{3}_{31} = \frac{1}{r} \\
&\Gamma^{1}_{22} = -\frac{r}{B} \hspace{1cm} ~\Gamma^{3}_{23} = \Gamma^{3}_{32} = \cot{\theta} \\
&\Gamma^{1}_{33} = -\frac{r\sen^2{\theta}}{B} \hspace{0.5cm} \Gamma^{2}_{33} = -\sen{\theta}\cos{\theta} \\
\end{split}
\end{equation}

donde las primas son derivaciones con respecto a $r$ y los puntos con respecto a $t$.

Luego a partir de la definición del tensor de Riemann (ec. \ref{riemann}) 
\begin{equation*}
R^\mu_{\alpha \nu \beta}\equiv\partial_{\nu}\Gamma^{\mu}_{\alpha \beta}-\partial_{\beta}\Gamma^{\mu}_{\alpha\nu}+\Gamma^{\mu}_{\nu\gamma}\Gamma^{\gamma}_{\alpha\beta}-\Gamma^{\mu}_{\beta\gamma}\Gamma^{\gamma}_{\alpha\nu}
\end{equation*}
y del tensor de Ricci (ec. \ref{ricci}) 
\begin{equation*}
R_{\alpha\beta} \equiv R^\mu_{\alpha \mu \beta}
\end{equation*}

Lo único que tenemos que hacer para obtener el tensor de Ricci en función de los símbolos de Christoffel es intercambiar los $\nu$ por $\mu$ y obtenemos
\begin{equation}
R^\mu_{\alpha \mu \beta}= R_{\alpha\beta}= \partial_{\mu}\Gamma^{\mu}_{\alpha \beta}-\partial_{\beta}\Gamma^{\mu}_{\alpha\mu}+\Gamma^{\mu}_{\mu\gamma}\Gamma^{\gamma}_{\alpha\beta}-\Gamma^{\mu}_{\beta\gamma}\Gamma^{\gamma}_{\alpha\mu}    
\end{equation}
y entonces los simbolos de Ricci son:
\begin{equation}
\begin{split}
R_{00} &= -\frac{A'}{4B}\biggr(\frac{A'}{A}+\frac{B'}{B}\biggr)+\frac{A''}{2B}+\frac{A'}{Br}-\frac{\Ddot{B}}{2Bc^2}+\frac{\dot{B}}{4Bc^2}\biggr(\frac{\dot{A}}{A}-\frac{\dot{B}}{B}\biggr) \\
R_{11} &= \frac{A'}{4A}\biggr(\frac{A'}{A}+\frac{B'}{B}\biggr)-\frac{A''}{2A}+\frac{B'}{Br}-\frac{\Ddot{B}}{2Ac^2}-\frac{\dot{B}}{4Ac^2}\biggr(\frac{\dot{A}}{A}-\frac{\dot{B}}{B}\biggr) \\
R_{22} &= -\frac{r}{2B}\biggr(\frac{A'}{A}-\frac{B'}{B}\biggr)-\frac{1}{B}+ 1 \\
R_{33} &= \biggr[-\frac{r}{2B}\biggr(\frac{A'}{A}-\frac{B'}{B}\biggr)-\frac{1}{B}+ 1 \biggr]\sin^2{\theta} = R_{22}\sin^2{\theta}\\
R_{01} &= R_{10} = \frac{\dot{B}}{Brc} \\
\end{split}
\end{equation}

\newthought{Hasta aquí} es lo máximo que podemos generalizar el campo gravitacional esféricamente simétrico. Para determinar $A$ y $B$ necesitaremos resolver el lado derecho de la ecuación de Einstein utilizando el tensor de energía-momento y luego comparar con el lado izquierdo. 

\subsection*{\textbf{Cálculo del Tensor de Estrés-Energía}}
En nuestro caso, solo hay campos eléctricos en el espacio, y sabemos de la simetría esférica del problema que el campo solo puede tener componente radial y que dicha componente solo no puede depender de $\phi$ o $\theta$, entonces
\begin{equation}
    E_1=E_r(t,r)=cF_{01}=-cF_{10}
\end{equation}
El tensor de campo electromagnético dado por (\ref{tensorcampoelectromagnetico}) queda como
\begin{equation}
    F_{\alpha\beta}=
    \begin{pmatrix}
    0 & E_r/c & 0 & 0\\
    -E_r/c & 0 & 0 & 0\\
    0 & 0 & 0 & 0\\
    0 & 0 & 0 & 0
    \end{pmatrix}
\end{equation}

Y ahora podemos usar (\ref{tensorestresdownstairs}) para calcular las componentes del tensor de estrés-energía. El primer término es
\begin{equation*}
\begin{split}
\frac{1}{4}g_{\alpha\beta}F_{\mu\nu}F^{\mu\nu} &= \frac{1}{4}g_{\alpha\beta}(F_{\mu 0}F^{\mu0} + F_{\mu 1}F^{\mu1})= \frac{1}{4}g_{\alpha\beta}(F_{00}F^{00} + F_{10}F^{10} + F_{0 1}F^{01}+ F_{11}F^{11}) \\
&= \frac{1}{2}g_{\alpha\beta}F_{0 1}F^{01}
\end{split}
\end{equation*}
donde sumamos sobre los índices repetidos, usamos que el tensor $F_{\alpha\beta}$ solo tiene componentes no nulas $F_{10}$ y $F_{01}$ y que es antisimétrico. El segundo término nos da
\begin{equation}
g_{\beta\nu}F_{\alpha\mu}F^{\nu\mu} = g_{\beta\nu}F_{\alpha 0}F^{\nu 0} + g_{\beta\nu}F_{\alpha 1}F^{\nu 1} = g_{\beta 1}F_{\alpha 0}F^{10} + g_{\beta 0}F_{\alpha 1}F^{01}
\end{equation}
y finalmente llegamos a una expresión para el tensor
\begin{equation}
T_{\alpha\beta} = \frac{1}{\mu_0}\biggr(\frac{1}{2}g_{\alpha\beta}F_{0 1}F^{01} - g_{\beta 1}F_{\alpha 0}F^{10} - g_{\beta 0}F_{\alpha 1}F^{01}\biggr)
\end{equation}
A partir de esto obtener las componentes del tensor fácil, simplemente reemplazamos $\alpha$ y $\beta$ por todas las combinaciones posibles entre $0$ y $3$ y reemplazamos las entradas del tensor $g_{\alpha\beta}$ según (\ref{tensoresmetricos}) y obtenemos que los únicos 4 términos no nulos 
\begin{equation}
\begin{split}
T_{00} &= \frac{1}{\mu_0}\biggr(\frac{1}{2}g_{00}F_{01}F^{01} - g_{00}F_{01}F^{01}\biggr)= -\frac{1}{2\mu_0}AF_{01}F^{01}\\
T_{11} &=\frac{1}{2\mu_0}B F_{01}F^{01}\\
T_{22} &= -\frac{1}{2\mu_0}r^2 F_{01}F^{01} \\
T_{33} &= \frac{1}{2\mu_0}r^2 \sen^2{\theta} F_{01}F^{01} = T_{22}\sen^2{\theta}
\end{split}
\label{componentestensorestresenergia}
\end{equation}

Ahora que tenemos ambos lados de la ecuación de Einstein estamos en condiciones de comparar el tensor de Ricci con el de energía-momento para poder finalmente calcular $A$ y $B$ y obtener nuestra métrica.

\subsection*{\textbf{Comparación tensor de Ricci y de Estrés-Energía}}

Es fácil, ver de la ecuación de campo que si una entrada en el tensor de Ricci es nula lo será en el de estrés-energía y viceversa. Por ejemplo, la entrada $T_{01}$ es nula entonces la entrada $R_{01}$ deberá serlo y de la ecuación para $R_{01}$ podemos concluir que
\begin{equation}
    \dot{B} = 0 
\end{equation}
$B$ no depende de $t$.

Por otro lado, de (\ref{componentestensorestresenergia}) podemos ver que se cumple
\begin{equation}
    \frac{T_{00}}{A} + \frac{T_{11}}{B} = 0 
\end{equation}
Como sabemos que cada coordenada de $R_{\alpha\beta}$ es una constante multiplicada por $T_{\alpha\beta}$ (ec. \ref{riccidownstairs}) podemos escribir
\begin{equation}
\frac{R_{00}}{A} + \frac{R_{11}}{B} = \frac{1}{rB}\biggr(\frac{A'}{A}+\frac{B'}{B}\biggr) = 0 
\end{equation}
lo cual implica que 
\begin{equation}
0 = \biggr(\frac{A'}{A}+\frac{B'}{B}\biggr) \Rightarrow  \frac{1}{AB}(A'B+AB') = \frac{\partial}{\partial r} \ln(AB)
\end{equation}
entonces $AB$ debe ser constante con respecto a $r$ y lo podemos escribir como 
\begin{equation}
    AB = f(t)
\end{equation}
La relación entre $F_{01}$ downstairs y $F^{01}$ upstairs está dada por
\begin{equation}
    F_{01} = g_{00}g_{11}F^{01} = -fF^{01}
\end{equation}
donde usamos que $g_{00}=A$ y $g_{11}=-B$ y que $AB=f$ 

En este punto necesitamos las ecuaciones de Maxwell en su forma tensorial
\begin{equation}
\begin{split}
\nabla_\beta F^{\alpha\beta} &= 0 \\
\nabla^\mu F^{\alpha\beta} + \nabla^\beta F^{\mu\alpha}  +\nabla^\alpha F^{\beta\mu} &= 0
\end{split}
\end{equation}

donde $\nabla_\gamma$ es la derivada covariante (ó gradiente absoluto) que ya hemos definido en el capítulo anterior.

\begin{marginfigure}
\begin{remarkbox}{Teorema de Birkhoff}
Para demostrar que la función $A$ no depende del tiempo, es posible apelar al siguiente argumento: como sabemos que $AB=f(t)$,  que $\dot{B}=0$, es necesario que $A$ tenga la siguiente forma:

$$A(r,t)=a(r)f(t)$$

Donde $a=k/B$. El coeficiente que acompaña al término $dt^2$ en la métrica será entonces $-Kf(t)g(r)$, de donde concluimos que $Kf(t)$ tiene que ser positivo (para que la \textit{signature} de la métrica nos asegure 3 dimesiones espaciales y una temporal). Esto significa que podemos redefinir la coordenada $t$ como:

\begin{equation}
    dt_{new}=dt_{old}\sqrt{Kf(t)}
\end{equation}

Es decir que la libertad a la hora de elegir la coordenada $t$ nos permite independizarnos del tiempo en la métrica alrededor de un objeto con simetría esférica. Esto es conocido como \textbf{Teorema de Birkhoff}, y si bien aplica a el espacio vacío, también puede generalizarse para incorporar las Ecuaciones de Maxwell en el llamado \textbf{Electrovacío de Reissner-Nordström}.
\end{remarkbox}
\end{marginfigure}


Luego la primera ecuación de Maxwell se escribe como
\begin{equation}
\nabla_\beta F^{\alpha\beta} = \partial_\beta F^{\alpha\beta} + \Gamma^\alpha_{\mu\beta}F^{\mu\beta} + \Gamma^\beta_{\mu\beta}F^{\alpha\mu} = 0 
\end{equation}
Si tomamos $\alpha=1$ en la anterior tenemos 
\begin{equation}
\partial_0 F^{10} + \Gamma^1_{\mu\beta}F^{\mu\beta} + \Gamma^\beta_{\mu\beta}F^{1\mu} = 0
\end{equation}
donde el termino con la derivada $\partial_0$ es el único que sobrevive por las entradas no nulas del $F_{\alpha\beta}$. Veamos la suma implícita en el segundo término
\begin{equation}
\Gamma^1_{\mu\beta}F^{\mu\beta} = \Gamma^1_{\mu0}F^{\mu0} +\Gamma^1_{\mu1}F^{\mu1}= \Gamma^1_{00}F^{00} + \Gamma^1_{10}F^{10} + \Gamma^1_{10}F^{10} + \Gamma^1_{11}F^{11}=0
\end{equation}
Luego este término es nulo porque $\Gamma^1_{01}=\Gamma^1_{10}=F^{00}=F^{11}=0$. Y el tercer término también es nulo 
\begin{equation}
\Gamma^\beta_{\mu\beta}F^{1\mu} = F^{10}(\Gamma^0_{00}+\Gamma^1_{01}+\Gamma^2_{02}+\Gamma^3_{03})=0
\end{equation}
donde los símbolos de Christoffel en el paréntesis se anulan, entonces llegamos a 

\begin{equation}
    \partial_0 F^{10} = 0
\end{equation}
Esto implica, que $F^{10}$ y por lo tanto $E(r)$ no dependen del tiempo y llegamos a 
\begin{equation}
    E_r = E_r(r)
\end{equation}

Si ahora tomamos de entrada el caso $\alpha=0$ 
\begin{equation}
\partial_1 F^{01} + \Gamma^0_{\mu\beta}F^{\mu\beta} + \Gamma^\beta_{\mu\beta}F^{0\mu} = 0
\end{equation}
Procediendo como antes podemos ver que el segundo término desaparece nuevamente pero el tercero no
\begin{equation}
\begin{split}
\Gamma^\beta_{\mu\beta} F^{0\mu} &= \Gamma^\beta_{1\beta} F^{01} = F^{01}(\Gamma^0_{10} +\Gamma^1_{11}+\Gamma^2_{12}+\Gamma^3_{13}) \\
&= F^{01}(\frac{A'}{2A} + \frac{B'}{2B} + \frac{2}{r}) = \frac{2}{r}F^{01} \Rightarrow \\
&\Rightarrow \frac{A'}{2A} + \frac{B'}{2B} = 0 \Rightarrow \frac{1}{2} \biggr(\frac{\partial}{\partial r}\ln(f)\biggr) = 0  
\end{split}
\end{equation}
porque sabíamos que $f(t)$ solo depende de $t$ y luego
\begin{equation}
    0 = \frac{\partial}{\partial r}F^{01} + \frac{2}{r}F^{01}
\end{equation}
Está última ecuación es fácil de resolver integrando 
\begin{equation}
    F^{01} = \frac{C}{r^2}
\end{equation}
donde $C$ es una constante de integración. Lo cual nos permite escribir
\begin{equation}
    E_r = \frac{C}{r^2}
\end{equation}
Podemos utilizar el teorema de Gauss en una superficie Gaussiana esférica y concluir que
\begin{equation}
    E_r = \frac{Q}{4\pi\epsilon_0 r^2}
    \label{Er}
\end{equation}
Esta es esencialmente la ley de Coulomb pero recordemos que $r$ es la circunferencia reducida que \textbf{no necesariamente} mide la distancia radial real en el espacio-tiempo que estamos analizando. 

Ahora solo nos falta expresar $A$ y $B$ como funciones de $r$ y terminamos. Partamos de la siguiente ecuación de Einstein
\begin{equation}
    R_{22} = \frac{8\pi G}{c^4} T_{22}
\end{equation}
Entonces
\begin{equation}
R_{22} = -\frac{r}{2B}\biggr(\frac{A'}{A} - \frac{B'}{B}\biggr) - \frac{1}{B}+1 = -\frac{1}{f}\frac{\partial}{\partial r}(rA) +1
\end{equation}
donde usamos que $f=AB$ y reglas de derivación. Ahora usamos la componente $T_{22}$ del tensor que ya calculamos (ec.\ref{componentestensorestresenergia}) y obtenemos
\begin{equation}
-\frac{1}{f} \frac{\partial}{\partial r}(rA) +1 = \frac{1}{f}\frac{8\pi G}{c^4} \frac{1}{2\mu_0 c^2} r^2 E_r^2
\end{equation}
Reemplazando $E_r$ por (ec. \ref{Er}) 
\begin{equation}
\frac{\partial}{\partial r} (rA) = f - \frac{GQ^2}{4\pi c^6 \mu_0 \epsilon_0^2r^2}
\end{equation}
Si ahora integramos y usamos que $c^2\mu_0= 1/\epsilon_0$ tenemos
\begin{equation}
    A = f + \frac{C_1(t)}{r} + \frac{GQ^2}{4\pi \epsilon_0 c^4 r^2}
    \label{A}
\end{equation}
Ahora, cuando $Q=0$, una de las suposiciones es que la métrica debe reducirse a la de Schwarzchild. Entonces, como mostramos en la ecuación (\ref{g00}), cuando $r$ es muy grande, entramos en límite de campo débil (Weak Field Limit) y $g_{00}$ tiende a 
\begin{equation*}
g_{00} = -1 - \frac{2GM}{c^2 r}
\end{equation*}
Entonces, en este límite, las geodésicas deben estar de acuerdo con el movimiento clásico gravitacional de Newton y mirando (ec. \ref{A}) se debe cumplir que $f=1$ y por lo tanto que 
\begin{equation}
C(t)= -\frac{2GM}{c^2} =- r_s
\end{equation}
que es el famoso \textbf{radio de Schwarzchild}. Además podemos definir 
\begin{equation}
    r_Q^2 = \frac{GQ^2}{4\pi\epsilon_0 c^4}
\end{equation}
y finalmente $A$ y $B$ quedan
\begin{equation}
\begin{split}
A &= 1 - \frac{r_s}{r} + \frac{r_Q^2}{r^2} \\
B &= \biggr(1 - \frac{r_s}{r} + \frac{r_Q^2}{r^2}\biggr)^{-1}
\end{split}
\end{equation}
y el tensor métrico es 

\begin{remarkbox}{Tensor métrico del espacio-tiempo de Reissner-Nordström}
\begin{equation*}
g_{\alpha\beta} = \begin{pmatrix}
\biggr(1 - \frac{r_s}{r} + \frac{r_Q^2}{r^2}\biggr) & 0 & 0 & 0 \\
0 & \biggr(1 - \frac{r_s}{r} + \frac{r_Q^2}{r^2}\biggr)^{-1} & 0 & 0\\
0 & 0 & -r^2 & 0 \\
0 & 0 & 0 & -r^2\sen^2{\theta} \\
\end{pmatrix}
\end{equation*}
\end{remarkbox}

\newthought{Hemos llegado} a la métrica de Reissner-Nordström completa derivada a partir de las ecuaciones de campo de Einstein junto con las ecuaciones de Maxwell. 


\subsection*{\textbf{Sobre las unidades geométricas}}
A partir de ahora, salvo que indiquemos lo contrario, usaremos las unidades geométricas. Esto es, tomaremos la velocidad de la luz $c$, y la constante de gravitación universal $G$ como :

\begin{remarkbox}{Consideración de las unidades geométricas}
\begin{equation*}
    c=G=1\ \ \textit{adimensional}
\end{equation*}
\end{remarkbox}

Hacemos esto para olvidarnos de las constantes, y hacer menos engorrosas las ecuaciones.
\begin{table}[h]
  \begin{center}
    \begin{tabular}{lccl}
      \toprule
      Variable & Unidades SI & Unidades Geom. & Factor \\
      \midrule
      Masa & $kg$ & $m$ &$c^2 G^{-1}$ \\
      Longitud & $m$ & $m$ & 1 \\
      Tiempo & $s$ & $m$ & $c^{-1}$\\
      Velocidad & $m s^{-1}$ & adim & $c$  \\
      Aceleración & $m s^{-2}$ & m{-1} & $c^2$  \\
      Fuerza & $kg m s^{-2}$ & adim & $c^4 G^{-1}$  \\
      Momento Angular & $kg m^2 s^{-1}$ & $m^2$ & $c^3 G^{-1}$ \\
      Momento & $kg m s^{-1}$ & $m$ & $c^3 G^{-1}$ \\
      Energía & $kg m^2 s^{-2}$ & $m$ & $c^{4} G^{-1}$\\
      Densidad de Energía & $kg m^{-1} s^{-2}$ & $m^{-2}$ & $c^4 G^{-1}$  \\
      \bottomrule
    \end{tabular}
  \end{center}
  \caption{Unidades Geométricas. Para convertir Geom. $\rightarrow$ SI, multiplicar por el factor. Para convertir SI $\rightarrow$ Geom., dividir por el factor. De forma general, para unidades SI de ''$kg^\alpha m^\beta s^\gamma$'', las unidades geométricas son ''$m^{\alpha + \beta + \gamma}$''.}
  \label{geounits}
\end{table}
\section{\huge{Geodésicas en la Geometría de Reissner-Nordström}}

\textcolor{myred}{\hrule}

\newthought{Ahora encontremos las ecuaciones que describen el movimiento de fotones y partículas no cargadas.} La partícula seguirá una geodésica \textbf{\textit{time-like}} mientras que el fotón una geodésica \textbf{\textit{light-like}}. Sea $x^\alpha = x^\alpha (\lambda)$ una curva parametrizada por $\lambda$, entonces debe cumplir la Ec. (\ref{geodesicx}):
\begin{equation}
    \frac{d^2 x^\alpha}{d \lambda^2} + \Gamma^\alpha_{\mu\nu} \frac{d x^\mu}{d \lambda} \frac{d x^\nu}{d \lambda} = 0
\end{equation}

Donde $\Gamma^\alpha_{\mu\nu}$ son los símbolos de Christoffel asociados a la métrica. Para una geodésica \textit{time-like} lo más natural es definir el parámetro $\lambda$ como el tiempo propio $\tau$, mientras que una geodésica \textit{light-like} no puede parametrizarse con $\tau$. Mantendremos por ahora el parámetro $\lambda$ para trabajar con los dos casos. Reemplazando los símbolos de Christoffel para nuestra métrica las ecuaciones de movimiento son:
 
\newthought{Para $\alpha=0$:}
\begin{equation}
    \frac{d^2 t}{d \lambda^2} + \frac{A^\prime}{A} \frac{d t}{d \lambda} \frac{d r}{d \lambda} = 0
\label{eqt3.1}
\end{equation}
\newthought{Para $\alpha=1$:}
\begin{equation}
\begin{split}
    \frac{d^2 r}{d \lambda^2} + \frac{A^\prime}{2B} \left(\frac{d t}{d \lambda}\right)^2 + \frac{B^\prime}{2B} \left(\frac{d r}{d \lambda}\right)^2 - \frac{r}{B} \left(\frac{d \theta}{d \lambda}\right)^2 &\\- \frac{r \sin^2{\theta}}{B} \left(\frac{d \phi}{d \lambda}\right)^2 &= 0
\label{eqradial3.1}
\end{split}
\end{equation}
\newthought{Para $\alpha=2$:}
\begin{equation}
    \frac{d^2 \theta}{d \lambda^2} + \frac{2}{r} \frac{d \theta}{d \lambda} \frac{d r}{d \lambda} - \sin{\theta}\cos{\theta} \left(\frac{d \phi}{d \lambda}\right)^2= 0
\end{equation}
\newthought{Para $\alpha=3$:}
\begin{equation}
    \frac{d^2 \phi}{d \lambda^2} + \frac{2}{r} \frac{d \phi}{d \lambda} \frac{d r}{d \lambda} - 2\cot{\theta}\frac{d \phi}{d \lambda} \frac{d \theta}{d \lambda}= 0
\label{eqazimutal3.1}
\end{equation}

Debido a la simetría esférica la trayectoria debe estar contenida en un plano definido por las condiciones iniciales\footnote{Trazamos el plano que contiene la velocidad en el instante inicial, ya sea de la partícula o del fotón. Sí la trayectoria se saliera de ese plano, la dirección en la que lo haga sería preferencial, rompiendo la simetría esférica del sistema.}, por lo que podemos poner sin ninguna pérdida de generalidad que $\theta=\pi/2$ en todo momento. Esto nos anula las derivadas de $\theta$, y las ecuaciones de movimiento se simplifican. Ahora, las Ecs. (\ref{eqradial3.1}) y (\ref{eqazimutal3.1}) nos quedan:

\begin{equation}
    \frac{d^2 r}{d \lambda^2} + \frac{A^\prime}{2B} \left(\frac{d t}{d \lambda}\right)^2 + \frac{B^\prime}{2B} \left(\frac{d r}{d \lambda}\right)^2 -  \frac{r}{B} \left(\frac{d \phi}{d \lambda}\right)^2 = 0
\label{eqradial3.2}
\end{equation}
\begin{equation}
    \frac{d^2 \phi}{d \lambda^2} + \frac{2}{r} \frac{d \phi}{d \lambda} \frac{d r}{d \lambda}= 0
\label{eqazimutal3.2}
\end{equation}

Sí dividimos la Ec. (\ref{eqazimutal3.2}) por $d\phi/d\lambda$, y usamos que:
\begin{equation}
\begin{split}
    \left(\frac{d\phi}{d\lambda}\right)^{-1} \frac{d^2\phi}{d\lambda^2} &= \frac{d}{d\lambda} \log{\left(\frac{d\phi}{d\lambda}\right)}\\
    \text{y, }\ \frac{2}{r} \frac{dr}{d\lambda} &= \frac{d}{d\lambda}\log{(r^2)}
\end{split}
\end{equation}
llegamos a que:

\begin{equation}
    \frac{d}{d\lambda}\log{\left(r^2\frac{d\phi}{d\lambda}\right)}=0 \Rightarrow r^2 \frac{d\phi}{d\lambda} = L = cte
\label{eqL}
\end{equation}

Donde $L$ es una constante del movimiento que coincide con el \textit{momento angular por unidad de masa} de la teoría Newtoniana. De manera similar, obtenemos para la Ec. (\ref{eqt3.1}) que:

\begin{equation}
    \frac{d}{d\lambda}\log{\left(A\frac{dt}{d\lambda}\right)}=0 \Rightarrow A \frac{dt}{d\lambda} = e = cte
\label{eqe}
\end{equation}

Con $e$ una constante del movimiento que se puede interpretar como la \textit{energía total relativista por unidad de masa}. Ahora usamos las Ecs. (\ref{eqL}) y (\ref{eqe}) en la Ec. (\ref{eqradial3.2}) y tenemos:

\begin{equation}
    \frac{d^2 r}{d \lambda^2} + \frac{A^\prime}{2B} \frac{e^2}{A^2} + \frac{B^\prime}{2B} \left(\frac{d r}{d \lambda}\right)^2 - \frac{r}{B} \frac{L^2}{r^4} = 0
\end{equation}
recordando que $B\prime=-A\prime/A^2$ y multiplicando por $2Bdr/d\lambda$:
\begin{equation}
\begin{split}
    0&=2B\frac{d^2 r}{d \lambda^2}\frac{dr}{d\lambda} - e^2 B^\prime \frac{dr}{d\lambda} + B^\prime \frac{dr}{d\lambda} \left(\frac{d r}{d \lambda}\right)^2 - \frac{2L^2}{r^3} \frac{dr}{d\lambda}\\
    &=\frac{d}{d\lambda}\left[ B \left(\frac{d r}{d \lambda}\right)^2 - e^2 B + \frac{L^2}{r^2}\right]
\end{split}
\end{equation}
por lo que:
\begin{equation}
    B \left(\frac{d r}{d \lambda}\right)^2 - e^2 B + \frac{L^2}{r^2} = -(e_0)^2 = cte
\label{eqe0}
\end{equation}

Donde $e_0$ se puede pensar como la \textit{energía total en reposo por unidad de masa}. Reescribiendo la Ec. (\ref{eqe0}) obtenemos:
\begin{equation}
    \left(\frac{d r}{d \lambda}\right)^2 = e^2 -A\left( \frac{L^2}{r^2} + {e_0}^2\right)
\label{eqradial3.3}
\end{equation}

Esta ecuación nos da $dr/d\lambda$ en función de $r$. Para obtener una ecuación de $dr/d\phi$ dividimos la Ec. (\ref{eqradial3.3}) por $(d\phi/d\lambda)^2 = L^2/r^4$:
\begin{equation}
    \left( \frac{dr}{d\phi} \right)^2 = \frac{r^4 e^2}{L^2} - r^2 A \left(1 + \frac{r^2 {e_0}^2}{L^2} \right)
\end{equation}
recordando que $A=1 - (r_s/r) + (r_Q/r)^2$, llegamos a:
\begin{equation}
\begin{split}
    \left( \frac{dr}{d\phi} \right)^2 = -{r_Q}^2 + r_s r - \left(1 + \frac{{r_Q}^2 {e_0}^2}{L^2} \right) r^2 &\\+ \frac{r_s {e_0}^2}{L_2} r^3 &- \frac{{e_0}^2 - e^2}{L^2} r^4
\end{split}
\label{eqrposta}
\end{equation}

Esta es la ecuación que buscamos, nos da $dr/d\phi$ en términos de $r$. En teoría ya con esto podemos encontrar la trayectoria de una partícula o fotón, tomando los valores de las constantes a partir de las condiciones iniciales. Sin embargo, la Ec. (\ref{eqrposta}) se puede simplificar aún más apelando a la métrica\footnote{Recordemos que $c=G=1$ y $\theta=\pi/2$.}:

\begin{equation}
    ds^2 = d\tau^2 = A dt^2 - \frac{1}{A} dr^2 - r^2 d\phi^2
\end{equation}

De las Ecs. (\ref{eqL}), (\ref{eqe}) y (\ref{eqe0}) tenemos:
\begin{equation}
\begin{split}
    d\phi^2 &= \frac{L^2}{r^4} d\lambda^2\\
    dt^2 &= \frac{e^2}{A^2} d\lambda^2\\
    dr^2 &= \left[e^2 - A\left({e_0}^2 + \frac{L^2}{r^2}\right)\right] d\lambda^2
\end{split}
\end{equation}

Sustituyendo $d\phi^2$, $dt^2$ y $dr^2$ en la métrica conseguimos:

\begin{remarkbox}{Relación entre $d\tau^2$ y $d\lambda^2$.}
\begin{equation}
    d\tau^2 = {e_0}^2 d\lambda^2
\label{doujou}
\end{equation}
\end{remarkbox}

Finalmente, sí tenemos una partícula con masa y parametrizamos con el tiempo propio, $d\lambda^2=d\tau^2$, según la Ec. (\ref{doujou}) tendremos que $e_0=1$. Sí tenemos un fotón el tiempo propio es nulo, $d\tau^2=0$, y por lo tanto $e_0=0$. Usando este resultado, junto a la Ec. (\ref{eqrposta}) escribimos al fin la ecuación de la trayectoria de un cuerpo no cargado:

\newthought{Para la luz:}
\begin{equation}
    \left( \frac{dr}{d\phi} \right)^2 = -{r_Q}^2 + r_s r - r^2 - \frac{e^2}{L^2} r^4
\label{eqlightlike}
\end{equation}
\newthought{Para un cuerpo, no cargado, con masa:}
\begin{equation}
    \left( \frac{dr}{d\phi} \right)^2 = -{r_Q}^2 + r_s r - \left(1 + \frac{{r_Q}^2}{L^2} \right) r^2 + \frac{r_s}{L_2} r^3 - \frac{1 - e^2}{L^2} r^4
\label{eqtimelike}
\end{equation}

\subsection*{\textbf{Trayectoria de un cuerpo cargado en la Geometría de R-N.}}
\newthought{Teniendo el espacio-tiempo \textit{lleno} de campo electromagnético, es natural preguntarse} la trayectoria de una carga. Para ello nos dirigimos al formalismo de Lagrange. Para una partícula cargada, en un espacio-tiempo cuya curvatura está descrita por $g_{\mu\nu}$, el lagrangiano es\footnote{Sí no hubiera campo electromagnético, $A_\mu=0$ y la solución se correspondería con el lagrangiano de una partícula bajo influencia de un campo gravitatorio. El término correspondiente a la ''energía potencial'' está incluido en el término con la métrica. En espacio plano, sin campos, se reduce a la energía cinética de una partícula en el vacío.}:

\begin{equation}
    \mathcal{L} = \frac{1}{2} g_{\mu\nu} \dot{x^\mu}\dot{x^\nu} + q A_\mu \dot{x^\mu}
\end{equation}

Donde $q$ es la \textit{carga por unidad de masa} de la partícula, $A_\mu$ es el cuadri-potencial electromagnético y los ''puntos'' representan una derivada respecto al tiempo propio $\tau$ de la partícula. En nuestro caso solo tenemos campo eléctrico radial, y el cuadri-potencial resulta ser:
  
\begin{equation}
    A_\mu=[A_0,A_1,A_2,A_3]=\left[\frac{Q}{4\pi\epsilon_0 r^2},\ 0,\ 0,\ 0 \right]
\end{equation}

Y, usando la métrica de Reissner-Nordström, el lagrangiano es:

\begin{equation}
    \mathcal{L} = \frac{1}{2}\left( A\dot{t}^2 - \frac{1}{A} \dot{r}^2 - r^2 \dot{\theta}^2 - r^2 \sin^2{\theta} \dot{\phi}^2 \right) + q \frac{Q}{4\pi\epsilon_0 r^2} \dot{t}
\end{equation}

Como antes la trayectoria será en un plano, así que podemos sin pérdida de generalidad considerar $\theta=\pi/2$ y $\dot{\theta}=0$. Luego, las ecuaciones de Euler-Lagrange son:

\newthought{Para $t$:}
\begin{equation}
\frac{d}{d\tau}\frac{\partial \mathcal{L}}{\partial \dot{t}} - \frac{\partial \mathcal{L}}{\partial t} = \frac{d}{d\tau}\left[ A\dot{t} + \frac{qQ}{4\pi\epsilon_0 r^2} \right]  = 0
\label{eqt3.4}
\end{equation}
\newthought{Para $r$:}
\begin{equation}
    \frac{d}{d\tau}\frac{\partial \mathcal{L}}{\partial \dot{r}} - \frac{\partial \mathcal{L}}{\partial r} = -\frac{\ddot{r}}{A} - \frac{A^\prime}{2} \dot{t}^2 - 2 r \dot{\phi}^2 + \frac{qQ}{2\pi\epsilon_0 r^3}  = 0
\label{eqradial3.4}
\end{equation}
\newthought{Para $\phi$:}
\begin{equation}
    \frac{d}{d\tau}\frac{\partial \mathcal{L}}{\partial \dot{\phi}} - \frac{\partial \mathcal{L}}{\partial \phi} =\frac{d}{d \tau}\left[ r^2 \dot{\phi} \right] = 0
\label{eqazimutal3.4}
\end{equation}

De las Ecs. (\ref{eqt3.4}) y (\ref{eqazimutal3.4}) tenemos las constantes de movimiento:

\begin{equation}
    A\dot{t} + \frac{qQ}{4\pi\epsilon_0 r} = e\ \ \text{, y }\ r^2 \dot{\phi} = L
\label{conservposta}
\end{equation}
la única diferencia de las constantes con el caso de la partícula sin carga, es el término $e$ que aparece la energía potencial electrostática $\frac{qQ}{4\pi\epsilon_0 r}$. Ahora, para llegar a una expresión de $dr/d\phi$ como función de $r$, necesitamos apelar a la métrica:

\begin{equation}
    ds^2=d\tau^2=Adt^2 - \frac{1}{A}dr^2- r^2d\phi^2
\end{equation}
Dividimos por $d\tau^2$, y multiplicamos por $A$:
\begin{equation}
    \dot{r}^2 + A - A^2 \dot{t}^2 + A r^2 \dot{\phi}^2 =0
\end{equation}

Usando las Ecs. (\ref{conservposta}), sustituimos $\dot{\phi}$ y $\dot{t}$:
\begin{equation}
    \dot{r}^2 + A - \left(e-\frac{qQ}{4\pi\epsilon_0 r}\right)^2 + A \frac{L^2}{r^2} =0
\end{equation}

Finalmente, dividiendo por $\dot{\phi}$:

\newthought{Para un cuerpo cargado con masa:}
\begin{equation}
\begin{split}
    \left( \frac{dr}{d\phi} \right)^2 &= -r^2 A\left(1+\frac{r^2}{L^2}\right) + \frac{r^4}{L^2}\left( e - \frac{qQ}{4\pi\epsilon_0 r} \right)^2\\
    &=-{r_Q}^2 + r_s r - \left( 1 + \frac{{r_Q}^2 - \left(\frac{qQ}{4\pi\epsilon_0}\right)^2}{L^2} \right) r^2 \\ &+ \frac{1}{L^2} \left(r_s - 2e\frac{qQ}{4\pi\epsilon_0}\right)r^3 - \frac{(1-e^2)}{L^2}r^4
\end{split}
\label{eqtimelikecargado}
\end{equation}

Esta es la ecuación que describe la órbita de una partícula cargada en el espacio-tiempo de Reissner-Nordström. Como era esperado, se reduce a la Ec. (\ref{eqtimelike}) cuándo hacemos $q=0$.